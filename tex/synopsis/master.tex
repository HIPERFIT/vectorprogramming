\documentclass[10pt,a4paper,final,oneside,openany,article]{memoir}


%BASIC PACKAGES
\usepackage[english]{babel} % last language decides document language!
\usepackage[utf8]{inputenc}          %text encoding
\usepackage[T1]{fontenc}
\usepackage[british]{isodate}

% Layout Fixes
\usepackage{booktabs} % nicer spacing between table rulers
\usepackage{microtype}
\usepackage{fixltx2e} % To prevent the figures from being placed
                      % out-of-order with respect to their
                      % "non-starred" counterparts


\usepackage{amsmath,amssymb, amsbsy}
\usepackage[amsmath,amsthm,thmmarks]{ntheorem}
\usepackage{semantic, stmaryrd}
\usepackage{graphicx}
\usepackage{todonotes}
\presetkeys{todonotes}{inline}{}


% Bibliography
%\usepackage[style=alphabetic,natbib=true]{biblatex}
\usepackage[style=numeric,natbib=true, defernumbers=true]{biblatex}

% Fonts
\usepackage{palatino}
\linespread{1.05}
\usepackage[scaled]{beramono}

% customize chapter pages
\makepagestyle{myheadings}
\makepagestyle{myheadingschapterpage}
\makeevenfoot{myheadingschapterpage}{}{\thepage}{}
\makeoddfoot{myheadingschapterpage}{}{\thepage}{}
\aliaspagestyle{chapter}{myheadingschapterpage}
\aliaspagestyle{title}{myheadingschapterpage}
\makeevenhead{myheadings}{}{%\scshape \thetitle
}{}
\makeoddhead{myheadings}{}{
%\footnotesize\scshape \thetitle%
}{}
\makeevenfoot{myheadings}{}{\thepage}{}
\makeoddfoot{myheadings}{}{\thepage}{}
\pagestyle{myheadings}

\def\thefigure{\arabic{figure}}
\setcounter{tocdepth}{0}
\setcounter{page}{2}


\setcounter{secnumdepth}{1}
\setcounter{chapter}{0}
\setsecheadstyle{\large\bfseries\raggedright}
\setsubsecheadstyle{\bfseries}



% Figures
\newsubfloat{figure}

\usepackage{sidecap}
\usepackage{caption}
\captionsetup{margin=0pt, font=small, labelfont=bf, format=hang}
\setlength{\abovecaptionskip}{0pt}
\setlength{\belowcaptionskip}{0pt}


% Default commands
\newcommand{\subimgwidth}{.48\textwidth}
\newcommand{\imgwidth}{.85\textwidth}

%
\theoremstyle{plain}
\theoremsymbol{\tiny $\Box$}
\newtheorem{Definition}[equation]{Definition}


%%% Local Variables: 
%%% mode: latex
%%% TeX-master: "master"
%%% End: 


\title{Vector languages for financial algorithms \\
      \small{Master thesis project description}
}
\author{
  Martin Dybdal -- \texttt{dybber@dybber.dk} -- \texttt{dpr964} \\
  Philip Lykke Carlsen -- \texttt{plcplc@gmail.com} -- \texttt{jhg363}
\\
University of Copenhagen}

\date{\today}

\bibliography{../bibliography/bibliography}


\begin{document}
\maketitle

\chapter{Introduction}
This is the project description of our joint master thesis work. The
master thesis will be supervised by Ken Friis Larsen and Cosmin
Oaneca.

\chapter{Motivation}
% What is the over all subject matter and why is it interesting to do
% further research in this subject
Finance is the subfield of economics that has to with valuing assets,
making investment decisions and measuring risk. The analysis used to
take decisions and performing risk management are complex and
computational intensive, what is more is that these problems has to be
solved in close to real time to still be relevant when done.

Many algorithms used for pricing financial contracts and performing
risk analysis are based on Monte Carlo simulation, a method of
approximating stochastic processes.

\todo{high performance computing in the financial sector, etc.}

The improvements in parallel vector languages and tools for general
purpose programming of GPUs will not only be applicable to financial
problems. High performance computing has a wide audience in the
scientific community, and improving the infrastructure for computation
will be useful in for example weather forecasting, bioinformatics and
simulation of chemical reactions\todo{name other applications}. The
real time requirements of certain financial applications are not that
important in these fields.


\chapter{Project goals}
% In short describe the experiment we are going to make
In this project we will first seek to find the extend to which
current parallel vector languages are applicable for the financial
domain, by implementing a set of predetermined pricing algorithms in
existing languages.

We will then determine how we can improve the state-of-the-art of
vector languages in the context of pricing algorithms. We will do so
in terms of expressiveness, safety and efficiency, by implementing
enhancements for one or more of the existing languages.

\chapter{Elaboration}
% Elaborate on the experiment, e.g. how it should be conducted and
% what others have done previously

\todo{mention GPU programming and highly parallel computer architecture}

As the first task in our project, we will conduct a survey among
existing vector languages, such as
Accelerate\cite{chakravarty2011accelerating}, Nikola
\cite{mainland2010nikola}, Repa \cite{keller2010regular}, Feldspar
\cite{axelsson2010feldspar}, Obsidian, Single-assignment C (SAC) and
\textsc{Nesl} (tentative list). These languages take different
approaches to leveraging the parallel computing power of contemporary
computer architectures, and have different limitations and advantages.

The survey will consist of writing a set of realistic example programs
from the financial domain, specifically option pricing
algorithms. Example algorithms includes \textit{Longstaff and
  Schwartz}, \textit{Black-Scholes}, \textit{Binomial option pricing}
and \textit{Brownian Bridge-method for Monte Carlo simulation}.


\todo{What limitations do we already know of, and which optimisations
  do we wish to implement: Memory coalescing, distributed vs. fusion
  of maps and fold, branch divergence, }

\chapter{Learning objectives}
% What are we (as students) going to learn from performing this
% project
After completing this master thesis we will be able to:
\begin{itemize}
%\item Explain the financial applications of the pricing algorithms
%  mentioned above, and the basics of financial contracts.
\item Describe the different types of financial products (options) and
  how option pricing is performed, with different classes of methods.
\item Describe current vector languages and present relevant criteria
  for their comparison and empirical evaluation.
\item Perform empirical evaluation of vector languages according to
  the criteria developed above,by implementing various pricing
  methods.
\item Write programs for GPUs and apply optimisations.
\item Implement extensions/enhancements of current vector languages.
\end{itemize}

% \chapter{Scope}
% % What are we NOT doing. That is, which limitations do we make
% \begin{itemize}
% \item 
% \end{itemize}

\newpage
\defbibheading{bibliography}{\chapter{References}} 
\printbibliography

%\printbibliography[keyword=dsls,heading=subbibliography,
%  title={Domain specific language infrastructure}]
\end{document}
