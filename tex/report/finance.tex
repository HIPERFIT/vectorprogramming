\chapter{Financial basics}
Finance is the subfield of economics that has to with valuing
assets. Assets can be material objects such as goods, buildings, staff
and so on, or immaterial objects such as knowledge, stocks, bonds,
patents, copyrights and so forth. In this thesis we will use some
algorithms for determining the value of \textit{options} as example
programs for illustrative purposes and benchmarks.

\section{Options}
Options are so called derivative financial constracts, that gives the
holder the right to buy or sell some underlying asset (e.g. stocks or
bonds) from another party at a pre-agreed price and time, but leaves
him the \textit{option} to leave be. If the holder decides to use this
right, we say that he \textit{exercises} the option. There are
different styles of options, the most common will be detailed below.

\subsection{European options}
An European option is the simplest of the common option types. The
holder of an European option can exercise the option at the pre-agreed
expiration date, and only at that point.

%European call option

%European put option


% \begin{tabular}[h!]{lll}
%                             & Issue date & Expiration date \\
% Market value of underlying  & \$10 & \$15 \\
% Paid by A                   & \$4 & \$10 \\
% \end{tabular}

\subsection{American options}
An American option is discern from European options by not having a
fixed exercise time, but a time interval of possible exercise times,
such that the holder may choose when to exercise. As with European
options, American options comes in both call and put variants.



% \subsection{Exotic options}
% Asian options
% Barrier options


%%% Local Variables:
%%% mode: latex
%%% TeX-master: "master"
%%% End:
