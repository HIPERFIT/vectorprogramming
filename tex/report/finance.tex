\chapter{Financial basics}
\todo{Better chapter title}

Finance is the subdiscipline of economics concerned with appraising
assets. Assets can be material objects such as goods, buildings, staff
and so on, or immaterial objects such as knowledge, stocks, bonds,
patents, copyrights and so forth. In this thesis we will use some
algorithms for determining the value of \textit{options} as example
programs for illustrative purposes and benchmarks.

\section{Options}
Options are so called derivative financial contracts, that gives the
holder the right to buy or sell some underlying asset (e.g. stocks or
bonds) from another party at a pre-agreed price and time, but leaves
him the \textit{option} to leave be. If the holder decides to use this
right, we say that he \textit{exercises} the option. 

\begin{example}[European Call Option]
  Imagine a stock valued at \$10 at time 0, and at time 1 its value
  might have gone up or down. Instead of directly trading this option,
  two parties might make an alternative agreement. The stock holder
  would issue an option saying:
  \begin{quotation}
    ``The holder of this option will at time 1 be eligible to buy the
    stock at price \$10 if he so desires''.
  \end{quotation}
  In this case the ``time 1'' is the \textbf{expiration time} of the
  option and ``\$10'' is the \textbf{strike price}. An interested
  buyer would then evaluate the changes of the option to have risen in
  value and might decide to buy the option for say ``\$2''. Then at
  time he gets to choose between loosing his ``\$2'' investment (not
  buying the stock) or buying the stock to the strike price
  ``\$10''. If the stock has increased more in value than the initial
  investment of ``\$2'', then the buyer would have benefited.
\end{example}
The option described in the above example is a \textit{European call
  option}.  There are other styles of options, the most common will be
detailed below.

\subsection{European options}
A European option is the simplest of the common option types. The
holder of a European option can exercise the option at the pre-agreed
expiration date, and only at that point. 

What makes the above option a \textit{call option} is the fact that
the option holder is eligible to buy the stock. Alternatively, a
\textit{put option} would give the holder the right to sell the stock.

\subsection{American options}
An American option is discern from European options by not having a
fixed exercise time, but a time interval of possible exercise times,
such that the holder may choose when to exercise. As with European
options, American options comes in both call and put variants.


% \subsection{Exotic options}
% Asian options
% Barrier options

\section{Option pricing}

% Black-Scholes

\subsection{Binomial pricing algorithm}


\subsection{Least Squares Monte Carlo algorithm}


%%% Local Variables:
%%% mode: latex
%%% TeX-master: "master"
%%% End:
