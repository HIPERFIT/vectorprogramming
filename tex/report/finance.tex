\chapter{Finance}
\todo{Better chapter title}

Finance is the subdiscipline of economics concerned with appraising
the value of assets. Assets can be material objects such as goods,
buildings, staff and so on, or immaterial objects such as knowledge,
stocks, bonds, patents, copyrights and so forth. In this thesis we
will use financial different algorithms for determining the value of
\textit{options} as example programs.

% for illustrative purposes and benchmarks.

\section{Options}
Options are so called derivative financial contracts, that gives the
holder the right to buy or sell some underlying asset (e.g. stocks or
bonds) from another party at a pre-agreed price and time, but leaves
him the \textit{option} to leave be. If the holder decides to use this
right, we say that he \textit{exercises} the option. The pre-agreed
price is called the \textit{strike price} and the agreed future time
point is called the \textit{expiration date}.

We make a distinction between the options that grants a right to buy
and those that grants a right to sell. \textit{Call options} grants a
right to buy some underlying asset, whereas \textit{put options}
grants a right to sell.

\begin{example}[European Call Option]
  Imagine a stock valued at \$10 at time 0, and at time 1 its value
  might have gone up or down. Instead of directly trading this option,
  two parties might make an alternative agreement. The stock holder
  would issue an option saying:
  \begin{quotation}
    ``The holder of this option will at time 1 be eligible to buy the
    stock at price \$100 if he so desires''.
  \end{quotation}
  In this case the ``time 1'' is the \textbf{expiration date} of the
  option and ``\$100'' is the \textbf{strike price}. An interested
  buyer would then evaluate the changes of the option to have risen in
  value and the two parties agree to buy the option for say
  ``\$2''. Then at time 1 he gets to choose between loosing his ``\$2''
  investment (not buying the stock) or buying the stock to the strike
  price ``\$100''. If the stock has increased more in value than the
  initial investment of ``\$2'', then the buyer would have benefited.
\end{example}

The option described in the above example is a \textit{European call
  option}.  There are other styles of options, and they mostly vary in
when it is possible to exercise them. A European option is the
simplest of the common option types. The holder of a European option
can exercise the option at the pre-agreed expiration date, and only at
that point. An American option discern from an European option by not
having a fixed exercise time, but a time interval of possible exercise
times, such that the holder may choose when to exercise in this
interval. This makes pricing more complicated as the choice \todo{...}

There are several other option types, which we will not detail much,
as we will focus on pricing algorithms for American options. They
typically vary in when the option can be exercised (\textit{Bermudan}
and \textit{Barrier options}) or how the strike price is determined
(\textit{Asian options}).

\section{Option pricing}

\subsection{Black-Scholes algorithm}
Black-Scholes is an algorithm for pricing European options.
\todo{Only write and keep this section if we really get Black-Scholes through as a benchmark}

\subsection{Binomial pricing algorithm}
One of the simplest algorithms for pricing American options is based
on a binomial probability model. Using this method, we assume that the
value of the underlying asset moves a fixed percentage up or down at
each time step. We usually assume that the probability of going up at
each time step corresponds to the riskless rate.

We can then represent all possible states from the current time to the
last possible exercise time as binary tree of up and down
movements. The root carries the current value, its two children
carries the value after one up or down movement, and so forth. The
leafs then represents the value at the last time step.

\subsection{Least Squares Monte Carlo algorithm}

\subsubsection{Sobol-sequence generation}

\subsubsection{Linear Least Squares solver}

%%% Local Variables:
%%% mode: latex
%%% TeX-master: "master"
%%% End:
