\part{Survey}
\chapter{Survey Setup}
In this part of our dissertation we present a small scale survey of
current vector languages. The goal was to orientate ourselves in the
current landscape of such languages and their
implementation. \todo{something about what others might use our
  results for -- it's not really only meants as an exercise for our
  benefit}

We have decided to evaluate and compare the languages Accelerate
\cite{}, Repa \cite{}, Nikola \cite{} and the \texttt{Data.Vector}
package\footnote{\url{http://hackage.haskell.org/package/vector}}. We
compare these four libraries to R programming language \cite{} and
CUDA, which are some of the languages currently used by financial
engineers. The R language is used for expressing financial algorithms
succintly, while CUDA is used for performance reasons, which is why we
have chosen to include both of these. Because of time constraints, we
have had to limit ourselves more than we originally had hoped for,
this means we have had to leave out languages such as Feldspar\cite{},
Obsidian\cite{}, Data Parallel Haskell\cite{}, Copperhead\cite{} and
NESL\cite{} from thesurvey, although they might have provided further
insigths.

\section{Programming language surveys}
To make a fair comparison, we have to be objective in our evaluation
of the languages. Obtaining objectivity in such a comparison is not
and easy task, though as it is hard to quantitatively measure aspects
such as the quality of language documentation (longer is not always
better, and it might be outdated). Also, as mentioned in (Lutz Prechelt, 2001):

\begin{quote}
  ``Any programming language comparison based on actual sample programs
  is valid only to the degree to which the capabilities of the
  respective programmers using these languages are similar.''
\end{quote}

We thus either have to acquire the same level of experience in all the
languages ourselves or find experts in each of the languages to do the
implementation. We have not had the resources to conduct a survey on
the same standard as Lutz Prechelt did in his paper ``Empirical
Comparison of Seven Programming Languages''. Here 10-20
implementations of the same algorithm was written by different people
(mostly graduate students) in each programming language under comparison.

\todo{How do we obtain this? Contacting Accelerate and Repa people to
  get their versions? We also have to document the development
  process, and that we couldn't write an efficient Accelerate version
  ourselves.}

\section{Comparison}

\paragraph{Qualitative questions}
\begin{itemize}
\item How hard was the installation process?
\item How portable is it?
\item Code review of language/compiler implementation.
\item Quality of documentation.
\end{itemize}
\paragraph{Quantitative}
\begin{itemize}
\item Benchmark of binomial pricer on expiry = 1,2,4,8,16,32,64 years.
\item Benchmark of Longstaff and Schwartz.
\item How many GHC extensions?
\item (Performance comparison between naive and optimized version of
  binomial pricer)\_F
\item (Levenshtein distance in abstract symbolic tokens)\_T between
  naive and optimized version of binomial pricer.
\item Version of GHC (how up to date is the code base)\_N
\item Project health (project activity, etc.)\_N
\item Reverse dependencies - how many packages
  depends on it?
\end{itemize}

\begin{itemize}
\item How do we carry out the survey?
%\item How are such surveys normally performed?
\item Which parameters do look for and measure?
\item How do we measure these things, e.g. how do we evaluate ``project health''?
%\item Why didn't we include some languages, but included others?
\end{itemize}

\chapter{Qualitative evaluation}
Evaluation of languages on qualitative parameters, such as ``How good
is the documentation?'', ``How tight is the correspondence to the
Rofls R-code?''

\chapter{Benchmarks}
Evaluation of languages on measurable parameters. Performance graphs,
number of dependencies, etc.

\chapter{Conclusion}
What are our recommended choice of languages for different use-cases?

%%% Local Variables:
%%% mode: latex
%%% TeX-master: "master"
%%% End:
