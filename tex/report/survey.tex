\part{Survey}
\chapter{Survey Setup}
In this and the following chapters we present a small-scale survey of
current vector languages. The survey was originally conducted to
orient ourselves in the current landscape of parallel functional
languages and their implementation, but the results might be of
interest for others. We do not know of other comparisons with similar
scope.

We have decided to evaluate and compare the languages Accelerate
\cite{chakravarty2011accelerating}, Repa \cite{keller2010regular},
Nikola \cite{mainland2010nikola} and the \texttt{Data.Vector}
package\footnote{\url{http://hackage.haskell.org/package/vector}}. We
compare these four libraries to the R programming language and NVIDIA's
CUDA platform, which are some of the languages currently used by
financial engineers. The R language is used for expressing financial
algorithms succinctly, while CUDA is used for performance reasons,
which is why we have chosen to include both of these. Because of time
constraints, we have had to limit ourselves more than we had originally
intended. This means we have had to leave out languages such as
Feldspar\cite{axelsson2010feldspar},
Obsidian\cite{svensson2011obsidian}, Data Parallel Haskell \cite{},
Copperhead\cite{Catanzaro2011} and NESL\cite{nesl} from the survey,
although they might have provided further insights.

\todo{Why only Haskell? Are there any languages we have left out
  entirely? What about Theano, ArBB, Qilin, Erlang, SaC, CnC-CUDA? Why
  aren't they here?}

\todo{}

\section{Programming language surveys}
To make a fair comparison, we have to be objective in our evaluation
of the languages. Obtaining objectivity in such a comparison is not
an easy task, as it is hard to quantitatively measure aspects such as
the quality of language documentation (longer is not always better,
and it might be outdated). Also, as mentioned by Lutz Prechelt in his
2001 paper ``Empirical Comparison of Seven Programming Languages'':

\begin{quote}
  ``Any programming language comparison based on actual sample programs
  is valid only to the degree to which the capabilities of the
  respective programmers using these languages are similar.''
\end{quote}

We would thus have to either acquire the same level of experience in all the
languages ourselves or find experts in each of the languages to do the
implementation. We have not had the resources to conduct a survey
following the standards set in the paper by Lutz Prechelt. For each
language under comparison he acquired 10-20 implementations of the
same algorithm from different programmers (mostly graduate
students). We could have set up a similar experiment by presenting a
programming challenge to the ``Haskell-Cafe''-mailing list and/or the
Haskell section on the Reddit website, and surely we could perhaps get
a decent benchmark in terms speed and memory usage for different
implementations by different developers, but we would not get answers
to qualitative questions about the development process. Another aspect
is that these languages are all in their early stages, and most people
we would find on those channels might be amateurs. We do not find
amateur work a good basis for an objective comparison.

\todo{could still be an interesting experiment to make in the future.}
\todo{look through the above argument again}

\section{Comparison}
From the survey we want to uncover three main questions about each
language, and it is from answers to these questions that we will make
a comparison. The three questions revolve around the health of
project, the expressiveness and ease of use of the language, and the
performance of the language. The following three sections will pose
these questions and describe how we have decided to test them.

\todo{Should we also include an evaluation on ``Performance
  \textit{guarantees}''? That is, an evaluation of the theoretical
  aspects and cost models?}

\subsection{Project Health}
How good is the project health? That is,
we want to determine how likely is it that development on the language
will continue and that it will keep getting funding and interest from
developers.

Subquestions include:
\begin{itemize}
\item Quality of documentation
\item Code review
\item Portability
\item Installation process
\item Number of dependencies (incl. GHC extensions)
\item Number of users (reverse dependencies)
\item Number of contributors
\item How easy is it to start contributing?
\item Latest project activity (e.g. which version of GHC does it compile on?)
\item Licensing
\item Funding
\end{itemize}

\subsection{Ease of use and expressiveness}

How easy is the language to use? And how expressive is it? Our
aim is to find a language that might be suitable for a financial
engineer and we thus want a suitably high-level language. It should
be near the complexity level of R. We still want the language to
expressive enough to cover the domain of financial algorithms.

Subquestions include
\begin{itemize}
\item Which programs can we write, and which can't we write?
  \begin{itemize}
  \item Can we write nested loops?
  \item Does it include all of: scans, folds, zip/map, stencils,
  \end{itemize}
\item How good are the error messages?
\item How high-level is it on the scale from ``R'' to ``CUDA''?
\end{itemize}

\subsection{Performance} How does the languages compare in a
performance benchmark?
\begin{itemize}
\item Benchmark of binomial pricer on expiry = 1,2,4,8,16,32,64,128 years.
\item Benchmark of Longstaff and Schwartz
\item How many in-code optimiser hints (inlining-statements, forcing
  of delayed arrays etc.) are necessary to get decent performance?
\item How does the performance of a naive implementation (no
  optimiser hints) compare to an optimised version?
\end{itemize}


%\begin{itemize}
%\item How do we carry out the survey?
%\item How are such surveys normally performed?
%\item Which parameters do look for and measure?
%\item How do we measure these things, e.g. how do we evaluate ``project health''?
%\item Why didn't we include some languages, but included others?
%\end{itemize}

\chapter{Qualitative evaluation}
Evaluation of languages on qualitative parameters, such as ``How good
is the documentation?'', ``How tight is the correspondence to the
Rofls R-code?''

\chapter{Benchmarks}
Evaluation of languages on measurable parameters. Performance graphs,
number of dependencies, etc.

\chapter{Conclusion}
What are our recommended choice of languages for different use-cases?

%%% Local Variables:
%%% mode: latex
%%% TeX-master: "master"
%%% End:
