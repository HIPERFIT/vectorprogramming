% Make space in the table of contents so the conclusion and
% bibliography is separate from the Language Extension-part
\addtocontents{toc}{\protect\vspace{10pt}}

\chapter{Conclusion}
\label{chap:Conclusion}
\section{Future Work}


\subsection{Nested data-parallelism on NVIDIA Kepler GPUs}
On newer GPUs supporting CUDA 5.0, such as the one we have had access
to, allows for 20 levels of nested calls, such that a kernel can spawn
other parallel kernels directly on the GPU, without returning to the
CPU. NVIDIA uses the term \emph{dynamic parallelism}\footnote{A short
  example is provided in
  \url{http://developer.download.nvidia.com/assets/cuda/docs/TechBrief_Dynamic_Parallelism_in_CUDA_v2.pdf}}

We only discovered this late in the project, and we have thus not
contemplated much on the idea of employing them for our
\lstinline{mapNest}-function. It should definitely be consider to use
new hardware capabilities for the problem, and it thus a thing that
would be worth looking closer at in the future.

\subsection{Prototype implementation of \texttt{sequential}}
Our idea of 'sequential' should have been introduced in previous
sections, but we should suggest implementing it more fully than we
have had time to and investigate which complications will arise, which
further extensions that are possible. All in all, we believe there is
a great potential in a \texttt{sequential}-construct, although we can
not do much more than speculate on the implementability and
practicality.

For further ideas for future work on \lstinline{sequential}, see
Chapter \ref{chap:directing-parallelism}.

\subsection{Broader scope of survey}
Many languages was left out in our survey, and only few parallization
patterns have been tested. It would thus be worth extending our survey
to cover other languages such as Theano, Feldspar, Obsidian, Bohrium
and Copperhead.

When it comes to algorithms implemented, we would like to implement
portfolio option pricers for both the binomial model and the LSM
algorithm, as they might giver better idea about the performance of
the languages for embarassingly parallel tasks.

On a final note about the future of our survey, we have during the
project found that this community of data-parallel languages, really
need a way of comparing their work, to make it more competitive. We
suggest that the community takes inspiration from the Computer
Languages Benchmark
Game\footnote{\url{http://benchmarksgame.alioth.debian.org/}}, and
creates a similar comparison between data-parallel languages where
results are updated regularly. There could be both a CPU and a GPU
category.

For further ideas for future work on \lstinline{sequential}, see
Chapter \ref{chap:directing-parallelism}.


% \item Cite SPL and the PJ/LexiFi paper and suggest using Longstaff and
%   Schwartz for pricing their contracts.
% \item Loop constructors for Nikola's \lstinline{P} monad (e.g. fold, see binomial example).
% \end{itemize}

\section{Conclusion}
\label{sec:conclusion}

See Section \ref{sec:conclusion}. $\bot$-tsssshhh.


%%% Local Variables:
%%% mode: latex
%%% TeX-master: "master"
%%% End:
