\chapter{Survey setup}
In this and the following chapters we present a small-scale survey of
a few current vector languages. The survey was originally conducted to
orient ourselves in the current landscape of parallel functional
languages and their implementation, but the results might be of
interest for others. We do not know of other comparisons with a
similar scope.

\section{Languages}
We have decided to evaluate and compare the languages Accelerate
\cite{chakravarty2011accelerating}, Repa \cite{keller2010regular},
Nikola \cite{mainland2010nikola} and the \texttt{Data.Vector}
package\footnote{\url{http://hackage.haskell.org/package/vector}}. We
compare these four libraries to the R programming language and
NVIDIA's CUDA platform, which are some of the languages currently used
by financial engineers
%\todo{cite - that both are used in finance}.
The R language is used for expressing financial algorithms
succinctly, while CUDA is used for performance reasons, which is why
we have chosen to include both of these. Because of time constraints,
we have had to limit ourselves more than we had originally
intended. This means we have had to leave out languages such as
Feldspar\cite{axelsson2010feldspar},
Obsidian\cite{svensson2011obsidian}, Data Parallel Haskell \cite{keller2010regular},
Copperhead\cite{Catanzaro2011} and NESL\cite{nesl} from the survey,
although they might have provided further insights. Also, as noted
previously, we have not been able to implement the complete Longstaff
and Schwartz algorithm in neither Nikola or Accelerate
\todo{make sure this is indeed written down somewhere in the previous chapter}.

%\todo{Why only Haskell? Are there any languages we have left out
%  entirely? What about Theano, ArBB, Qilin, Erlang, SaC, CnC-CUDA? Why
%  aren't they here?}

\section{Programming language surveys}
To make a fair comparison, we have to be objective in our evaluation
of the languages. Obtaining objectivity in such a comparison is not
an easy task, as it is hard to quantitatively measure aspects such as
the quality of language documentation (longer is not always better,
and it might be outdated). Also, as mentioned by Lutz Prechelt in his
2001 paper ``Empirical Comparison of Seven Programming Languages'':

\begin{quote}
  ``Any programming language comparison based on actual sample programs
  is valid only to the degree to which the capabilities of the
  respective programmers using these languages are similar.''
\end{quote}

We would thus have to either acquire the same level of experience in all the
languages ourselves or find experts in each of the languages to do the
implementation. We have not had the resources to conduct a survey
following the standards set in the paper by Lutz Prechelt. For each
language under comparison he acquired 10-20 implementations of the
same algorithm from different programmers (mostly graduate
students). We could have set up a similar experiment by presenting a
programming challenge to the ``Haskell-Cafe''-mailing list and/or the
Haskell section on the Reddit website, and surely we could perhaps get
a decent benchmark in terms speed and memory usage for different
implementations by different developers, but we would not get answers
to qualitative questions about the development process. Another aspect
is that these languages are all in their early stages, and most people
we would find on those channels might be amateurs. We do not find
amateur work a good basis for an objective comparison.

%\todo{could still be an interesting experiment to make in the future.}
%\todo{look through the above argument again}

\section{Comparison Metrics}
From the survey we want to uncover three main questions about each
language, and it is from answers to these questions that we will make
a comparison. The three questions revolve around the health of
the project, the expressiveness and ease of use of the language, and the
performance of the language. The following three sections will pose
these questions and describe how we have decided to test them.

\subsection{Project Health}

How good is the project health? That is, we want to determine how likely is it
that development on the language will continue and that it will keep getting
funding and interest from developers.

%%% Local Variables:
%%% mode: latex
%%% TeX-master: "../master"
%%% End:
