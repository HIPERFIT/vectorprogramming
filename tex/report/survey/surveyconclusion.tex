\chapter{Survey Conclusion}
In this chapter we summarise the conclusions made in the above
summary, and make suggestions for further studies. In Part
\ref{part:extensions} of this dissertation we will look closer on a
couple of the mentioned suggestions, and remaining can be seen as
future work. Additional future work is presented in Chapter
\ref{chap:Conclusion}.

\section{Language comparison and recommendations}
\todo{Which language to suggest for further studies: Nikola}
\todo{Which language to suggest for application development}

\section{Unfold}
\todo{Required by several of our cases, as neither Nikola nor
  Accelerate has primitives for iterative construction of arrays}
\todo{Nikola and Accelerate can only construct arrays where there are
  no dependencies between elements, through \texttt{fromFunction} and
  \texttt{generate}}

\section{Nesting}
\todo{We have shown how Accelerate and Nikola breaks down in
  cumbersome notation certain cases, where algorithms could be written
  clearly if nested layers of array operations was allowed} 

\todo{We argue that it is worth incorporating nesting in one way or
  another, and we will go into a possible way in the following
  sections}

\subsection{Directing parallelism}
\todo{Introducing nesting also requires some way to direct how
  parallelism will come about. Parallelism can be present at several
  layers, or vectorisation could be employed.}

\section{Where to synchronize}

\section{Memory reusage}

\section{Interfacing with existing libraries}

\section{Haskell infrastructure}
\label{sec:haskell_infrastructure}

%%% Local Variables:
%%% mode: latex
%%% TeX-master: "../master"
%%% End:
