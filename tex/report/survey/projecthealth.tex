\chapter{Evaluation of project health}
\label{chap:project-health}

% * Define project health
%  * How likely is it that development will continue and bugs will be fixed?

As mentioned in previously (see Chapter
\ref{cha:survey-setup}) we define the health of a project as
how likely it is that development will continue, whether programming
interfaces are subject to large change and properly documented.

% * Motivation for at undersøge det
%   * We take the stance of a library user, which has to decide on
%     a library, which seems they would be continuously maintained, such
%     that derivative works would be okay
%   * Not very important for the rest of our project, as the state of
%     the project does not make it unsuitable for research purposes.

We take the stance of a programming language user, which has to select
a data-parallel language with which he will base derivative work. The
goals of this chapter is not important for the rest of our project, as
bad project health does not make a programming language an unsuitable
subject for research.

In the following section we will look closer at the portability,
documentation quality, installation process and stability of Repa,
\texttt{Data.Vector}, Accelerate and Nikola.

\section{Portability}
% * Portability 
  
%   * As GHC is the defacto standard for Haskell developers, we have
%     only tested the libraries on this compiler.

%   * Nikola and Accelerate of course requires an NVIDIA CUDA graphics
%     card. OpenCL would be nice

Shared by all four languages/libraries is that they were developed for
Glasgow Haskell Compiler, which currently is the defactor stsandard
for Haskell development. We have only tested the libraries on this
compiler and we suspect none of them will run else where because they
rely on numerous GHC extensions (see Table
\ref{tab:dependency_status}).

Nikola and Accelerate only runs on NVIDIA GPUs as they rely on CUDA as
intermediate language in the compilation process. An incomplete
Accelerate backend for the more widely implemented OpenCL standard,
has been in developed by one of the authors of this master's thesis
(see \cite{dybdal2011opencl}), and has not been maintained for quite
a while.

We have collected a number of additional statistics about dependencies
in Table \ref{tab:dependency_status}, which we will let the reader
interpret as he wishes.


\begin{table}
  \centering
\begin{adjustbox}{minipage=1.3\textwidth,margin=0pt \smallskipamount,center}
  \begin{tabular}{l|rrllllr}
    Language    & Dependencies & Reverse dependencies & GHC extensions & GHC version \\ \hline
    Accelerate  & 5 (+18)      & 2                    & 19 (+9)        & 7.6.1 \\
    Nikola      & 21 0         & N/A                  & 26             & 7.4.2 \\
    Repa        & 6            & 11                   & 20             & 7.6.1 \\
    Data.Vector & 4            & 150+                 & 15             & 7.6.1 \\
  \end{tabular}
\caption{Status on language dependencies. Reverse dependencies were counted on Hackage in October 2012. 
  Nikola does not yet have a Hackage-release yet, why we could not count any reverse dependencies.
  When counting dependencies for Accelerate, there is one number for 
  the \texttt{accelerate} package, the frontend, and an additional number for the CUDA backend.}
\end{adjustbox}
  \label{tab:dependency_status}
\end{table}


\section{Documentation}
Repa, Accelerate and \texttt{Data.Vector} all have excellent
documentation available on Hackage and both Repa and Accelerate's
internals are documented in research papers
\cite{chakravarty2011accelerating}\todo[noinline]{cite repa papers}. We have not
been able to find information much documentation about
\texttt{Data.Vector} internals, which would have been helpful in
analysing its performance behavior.

Nikola does not provides almost no documentation, the most helpful
resource is the research paper from 2010 \cite{mainland2010nikola},
which is a bit outdated, as Nikola has undergone a major revision last
year. Geoffrey responded quickly on any questions we sent him, which
compensated a bit for the lack of documentation.

\section{Installation process}
% * Installation issues

%   * Nothing to note for Repa and Data.Vector
%   * Accelerate CUDA backend, horrible problems, see below
%   * Nikola ?!?!

\todo{write this section, notes below}


Accelerate and Nikola initially had contradicting dependencies, as
Accelerate had to run on GHC 7.6.1 and Nikola on GHC 7.4.2, and a few
additional dependencies was also incompatible. Luckily, it is possible
to install several versions of GHC side-by-side\footnote{We have used
  the \texttt{hsenv} package with good results. Our fork of
  \texttt{hsenv} is available at
  \url{https://github.com/dybber/hsenv}}. We are not sure if the
status is still the same, but our experience installing boths packages
-- and our failure to get the Feldspar library in a runnable state in
decent time -- examplified a large problem with the core Haskell
infrastructure. Many packages are not updated when newer versions of
their dependencies are released or new GHC releases are made. We will
thus suggest a change of policy on the Haskell Package Archive,
Hackage, to make sure that policies are updated or made unavailable
for installation through Cabal (see Section
\ref{sec:haskell_infrastructure}).


\section{Stability}
% * The two senses of the word: Likely to change and Likely to break in use
% * Accelerate CUDA backend seems very unstable in use. Language
%   interface seems stabilized.
% * Nikola seems very likely to change: 1 contributor, huge change
%   last year
% * Repa and Vector stable in both aspects. More dependencies,
%   less cutting edge.

There are two senses of the word stability that we want to evaluate,
we both want to evaluate whether the projects are \emph{likely to
  change} and whether they are \emph{likely to break in use}
(i.e. runtime errors).

Repa, Vector and Accelerate are all maintained at the University of
New South Wales, and work on all projects seems active, with regular
commits. We thus believe that work will continue. All three seems to
have stabilized on a frontend (unlikely to change), though backend
work on especially the Accelerate CUDA backend is still work in
progress, and is thus unstable (likely to break) and we have
encountered several bugs and installation issues with the CUDA
backend.

Nikola is maintained by a single author and is not yet released on
Hackage. It thus receives little exposure and we've been unable to
evaluate the number of users (which is thus potentially zero). There
currently is not any active development activity. Also, there is almost
no documentation, and we had to dig into the code to discover the
capabilities and limitations. From a users perspective it thus not a
wise selection for an implementation project, but that does not make
it unsuitable for a research project.


\begin{table}
  \centering
\begin{adjustbox}{minipage=1.1\textwidth,margin=0pt \smallskipamount,center}
  \begin{tabular}{l|rrllllr}
    Language    & Project age & Latest release & License & Contributors \\ \hline
    Accelerate  & 3-4 years   & June 2012      & BSD3    & 3 \\
    Nikola      & 1-2 years   & Not released   & Harvard (Berkeley style) & 1 \\
    Repa        & 1-2 years   & October 2012   & BSD3    & 4 \\
    Data.Vector & 4 years     & October 2012   & BSD3    & 9 \\
  \end{tabular}
  \caption{Project status. The number of contributors was acquired by
    inspection of commit logs. The data is from October 2012.}
\end{adjustbox}
\label{tab:project_status}
\end{table}
\todo{Perhaps update the above table. Project age could be given as absolute value (first commit).}

% * Activity

%   * Repa, Vector and Accelerate are updated regularly, while it seems
%     that development on Nikola is currently stagnated, until Geoffrey
%     Mainlan again finds time.



%%%%%%%%%%%%%%%%%%%%%%%%%%%%%%%%%%%%%%%%%%%%%%%%%%%%%%%%%%%%%%%%%%%%%%%%%%
% Everything below this point should be seen as notes, not finished text %
%%%%%%%%%%%%%%%%%%%%%%%%%%%%%%%%%%%%%%%%%%%%%%%%%%%%%%%%%%%%%%%%%%%%%%%%%%

 % \todo{Document the procedures and scripts that were used to collect reverse dependencies and to carry out installs}


% \section{Accelerate}
% \paragraph{Documentation.} The documentation for Accelerate resides mainly in
% two different places: The haddock API-docs on hackage, and more high-level
% documentation on the project's github wiki.  Other documentation reside in
% somewhat scattered web pages, most notably in the paper
% \cite{chakravarty2011accelerating}.  As of this writing, the main introductory
% material seems to be the github project wiki and the paper
% \cite{chakravarty2011accelerating}. Of these two, the wiki pages are more
% oriented towards building applications using Accelerate, while the paper
% focuses on detailing the current implementation and only briefly introduces the
% Accelerate language in an academic fashion.

% Unfortunately, many of the wiki pages are rather incomplete, so we must
% conclude that there has yet to be published any comprehensive introductory
% material on programming using Accelerate.
% The API reference documentation however is extensive.

% \paragraph{Installation process} We succeeded in getting Accelerate to run on
% GHC 7.4.2, but it still doesn't support GHC 7.6.1, possibly because of external
% dependencies.  We had some trouble installing the version on hackage, because
% of a minor bug in the cuda-bindings. The bug was caused by code generated by
% c2hs, so it might be something that depends on the version of CUDA installed.

% We had major trouble installing the development-version from their repository,
% mainly caused by a lot of dependency issues and conflicts. We still have to
% compile the CUDA-backend.  \todo{Text above is copied verbatim from wiki.}

% \section{Repa}

% \paragraph{Documentation.} Repa is extensively documented through both API
% documentation on hackage, various papers, tutorials and example programs, eg in
% \cite{lippmeier2012guiding} and \cite{keller2010regular}.  Everything is easily
% accessible directly from the front page of the repa homepage
% \cite{homepage:repa}.

% \section{Nikola}

% \paragraph{Documentation.} As Nikola is not published on Hackage, there is no
% automatically searchable generated documentation, and one has to manually
% generate haddock documentation. Counting source lines reveals a figure of 24\%
% comment lines, which is above the average comment ratio according to the
% ohloh.net open source project visualisation website\footnote{14-11-2012:
% ``Across all Haskell projects on Ohloh, 17\% of all source code lines are
% comments. For nikola-haskell, this figure is 24\%.''}. This figure turned out
% to be quite useless in this case, as we discovered most comments were just
% disabled code rather than documentation. Further, Nikola is described in the
% paper \cite{mainland2010nikola}.

% \paragraph{Installation process} Nikola requires running a classical style
% automake "configure"-script to produce the binding to the cuda-backend.

% We had a sizable amount of trouble executing the installation. \todo{why
% exactly? (Difficult to link with CUDA sdk, cabal dependency hell)}

% \section{Vector}

% \paragraph{Documentation.} The Vector package has extensive api-documentation
% available. Furthermore there is a tutorial on
% \cite{homepage:haskell:vectortutorial}.

% \subsection{Maintainance} Vector has seen regular maintainance as of October 2012.


%%% Local Variables:
%%% mode: latex
%%% TeX-master: "../master"
%%% End:
