\chapter{Evaluation of expressiveness}
\label{chap:expressiveness}
%\todo{Introduce the contents of this chapter}
%\todo{Can we find a more precise heading?} - Nope. Not now at least.

In this chapter we discuss and compare the expressiveness of each of the
languages we are considering.  The property of programming language
expressiveness is traditionally associated only with how beautiful and simple
encodings a set of algorithms lend themselves to. But since we are dealing with
a domain of algorithms where high computer performance is prerequisite for
acceptability, the need for control over the management of resources invariably
enters the picture -- at least with current programming technology. In this
survey therefore, resource management features of a programming language may
contribute to the lanugages expressivity.

We evaluate expressiveness of a lanugage by discussing its general features,
and by discussing noteworthy issues we have encountered while implementing our
benchmarks in the evaluated languages.

Although we already know that our eventual choice of language to extend stands
only between Nikola and Accelerate, here we also include CUDA, Repa,
Data.Vector and R to give a better perspective.

\section{CUDA/C}
CUDA/C is close to a one-to-one mapping of hardware capabilities to language
primitives, thus it is possible, though not necessarily easy, to implement anything
that the hardware is capable of running.

Every single resource must be explicitly managed -- only in CUDA version 5.0 is
there even a function call stack, supporting only a maximum of 20 levels of
nesting.  While this manual resource management enables the coding of very fast
algorithms, programs are also more tightly coupled with the parameters of the
hardware, and to assumptions about input size. So even though a CUDA programmer
has access to highly optimised linear algebra libraries such as CUBLAS
\cite{CUBLAS2013}, optimisations such as loop fusion must be coded by hand,
which rules out the use of such libraries. This optimisation often has profound
effects on performance \cite{mainlandhaskell}.

Implementing Algorithm \ref{alg:cuda-binom}, and adapting it to use just a
single option proved to be a substantial and error prone effort for us. This we
attribute to both that programming in CUDA is relatively foreign to us and the
lack of conventional debugging tools, such as a single-step debugger or a
machine simulator. The concrete reason for our trouble was almost always index
miscalculations in for-loops rather than misunderstandings of the general
programming model.

\section{R}

The R programming language is designed as a tool for exploring data sets using
statistics and plotting, and for coding prototypes of data analysis programs.
It is safe to say that the focus of R lies mainly on expressivity in the domain
of statistics and data analysis.

R is a very dynamic language, complete with anonymous functions, general
recursion and an immense library of high level operations. Standard R programs
execute on the CPU in their entirety. There is however an effort to utilise
GPUs in R to some extent through the \texttt{gputools}
package\cite{crangputools}.

R has become popular in scientific fields that are not primarily centered on
computer programming, such as biology and statistics
%\todo{cite}.
That in itself is good testament to the practical usefulnes of the language.

As a result of the extensive library support for this domain in R, the case
benchmarks we use in this language were able to offload much work to external
linear algebra routines.

\section{Data.Vector}

The \texttt{Data.Vector} library augment Haskell with a high performance
implementation of one-dimensional vectors. Vectors come in various flavours
that reside in different module namespaces. There are mutable vectors, which
are only usable from inside the IO or ST monads and must be allocated
explicitly, and there are immutable vectors, which are usable in pure
functional code. There is also a version of vectors working on unboxed values.

This vector library interfaces with the GHC compiler to provide the array
fusion optimisation. Taken together with the use of unboxed vectors one is able
to remove a lot of overhead associated with array and thunk construction, even
from ideomatic, functional Haskell code.

\section{Repa}

Repa is a library that attempts to bring high performance data parallelism to
Haskell by means of threaded multicore CPU parallelism. As a result, Repa
features does not impose more limits on the expressivity of Haskell in general.

The main contribution of Repa towards this end is that arrays in Repa are
parametrised by their internal representation and shape in the type:
\texttt{Array r sh a}.  This enables using different algorithms for processing
different array representations, and restricting certain operations for certain
representations. For example we have delayed arrays, which are simply
represented by an index domain and a function that maps indices to values.
These arrays support indexing, but each index operation will pay the cost of
computing the function.

Repa provides two main operations for manifesting arrays to memory:
\texttt{computeS} for sequential and \texttt{computeP} for parallel array
manifestation. While this division of execution modes is simple, it opens the
potential for expressing nested parallelism. In Repa, nested calls to
\texttt{computeP} is considered an error. A warning is issued on the terminal,
and the documentation says to expect reduced performance.

A consequence of this is that a hypothetical library using Repa that require
the use of array manifestation internally must provide separate parallel and
sequential versions, as a client program cannot use only the parallel version
if it intends to parallise on a another level of the program.

\section{Accelerate}
\label{sec:language-discussion-accelerate}
Accelerate provides a variety of array operations: Both scans,
segmented scans, folds, permutations, maps and zips, implemented as
parallel algorithmic skeletons. Arrays may be multidimensional,
denoted by a type variable in the same style as Repa.

Reductions (folds) are defined on arrays of arbitrary rank by
performing the reduction on the innermost dimension, yielding an array
with rank one less, or by folding all dimensions into a single scalar
value.  Maps are all elementwise regardless of the shape of the input
array, and scans are only defined on one-dimensional arrays.

Accelerate is characterised by a clean division between the frontend and the
various backends that exist. The Accelerate language is thus completely backend
agnostic, and backends simply export a function such as \hbox{\texttt{run :: Acc a ->
a}.}

Accelerate employs a meticulus partitioning of functions on arrays and functions
on scalar values. Array functions are all embedded in the \texttt{Acc a} type,
while scalars are embedded in the \texttt{Exp a} type. So, everything that is
capable of reducing an array or producing an array is carefully placed inside
\texttt{Acc}, and the functions usable for elementwise computation must
reside in \texttt{Exp}. Thus, the lack of nested parallelism is directly
encoded in the type system.

While this restriction probably makes it easy to ensure efficient execution of
the individual contructs, it impairs the composability of the language
constructs significantly, as the programmer needs to manually transform
algorithms with a naturally nested definition into something that will fit into
the view of Accelerate.

To illustrate this problem, consider the following small example.  While the example may seem a bit
abstract, it is directly derived from a problem we actually encountered while exploring
the Sobol sequence generation case, depicted in Algorithm
\ref{alg:sobol-inductive}. The actual code is included in Appendix
\ref{appendix:accelerate-sobol}.

Suppose we have a function \texttt{f}, defined for normal Haskell lists for notational clarity,
using the functions \texttt{k} and \texttt{h}:

\begin{verbatim}
f :: [Int] -> Int -> [Int]
f xs i = zipWith k xs (h i)
  where
    k :: Int -> Int
    h :: Int -> [Int]
\end{verbatim}

Now, if we want to apply \texttt{f} to a range of different \texttt{i}'s, what
we would reasonably do in Haskell would be just:

\begin{verbatim}
g :: [Int] -> [Int] -> [[Int]]
g xs ns = map (f xs) ns
\end{verbatim}

And for illustration, depictions of \texttt{f} and \texttt{g} as mathematical
expressions. Each application of \texttt{f} yields a row of the matrix:

\[
xs =
\begin{bmatrix}
  xs_0 \\
\vdots \\
xs_n
\end{bmatrix}
\qquad\qquad
ns =
\begin{bmatrix}
  ns_0 \\
\vdots \\
ns_m \\
\end{bmatrix}
\]
\[
f\ \ xs\ \ i =
\begin{bmatrix}
  k\ \ xs_0\ \ hi_0 \\
\vdots \\
  k\ \ xs_n\ \ hi_n \\
\end{bmatrix}
\quad
g\ \ xs\ \ ns =
\begin{bmatrix}
  f\ \left[xs_0 \ldots xs_n\right]\ ns_0 \\
\vdots \\
f\ \left[xs_0 \ldots xs_n\right]\ ns_m \\
\end{bmatrix}
\]

Now we want to port \texttt{f} and \texttt{g} to Accelerate.
For \texttt{f} there seems to be an obvious translation:

\begin{verbatim}
f :: Acc (Vector Int) -> Exp Int -> Acc (Vector Int)
f xs i = zipWith k xs (h i)
  where
    k :: Exp Int -> Exp Int
    h :: Exp Int -> Acc (Vector Int)
\end{verbatim}

But due to the lack of nested vectors, for \texttt{g} we must resort to a two-dimensional
array. And due to the restriction that mapped functions must reside in
\texttt{Exp}, we cannot simply do with just: \verb|map f (xs) ns|.

Instead we must distribute the map into the definition of \texttt{f} and
replicate \texttt{xs} to fit. But then we are left with an identical problem
relating to function \texttt{h}!. Now, assuming we succeed and denominate the
vectorised \texttt{h} function as
\verb|h' :: Acc (Vector Int) -> Acc (Array DIM2 Int)|,
here is what we arrived at for \verb|g'|:

\begin{verbatim}
g' :: Acc (Vector Int) -> Acc (Vector Int) -> Acc (Array DIM2 Int)
g' xs ns = let ns' = h' ns
              Z :. m :. n = unlift $ shape ns'
              xs' = replicate (Z :. m :. All) xs
          in zipWith k xs' ns'
\end{verbatim}

The transformation is illustrated below.

\[
\textrm{ns'} =
\begin{bmatrix}
  h(ns_0) \\
  \vdots \\
  h(ns_m)
\end{bmatrix}
\qquad
\textrm{xs'} =
\begin{bmatrix}
  xs_0 & xs_1 \ldots & xs_n \\
  \vdots & \ddots \\
  xs_0 & xs_1 \ldots & xs_n \\
\end{bmatrix}
\]
\[
g'\ \ xs\ \ ns =
\begin{bmatrix}
  k\ \ xs_0\ \ ns'_{0,0} & k\ \ xs_1\ \ ns'_{0,1} & \ldots &  k\ \ xs_n\ \ ns'_{0,n}  \\
\vdots & \ddots \\
  k\ \ xs_0\ \ ns'_{m,0} & k\ \ xs_1\ \ ns'_{m,1}  & \ldots &  k\ \ xs_n\ \ ns'_{m,n}   \\
\end{bmatrix}
\]

Using the above notation it is relatively easy to verify that \texttt{g} for
lists and \texttt{g'} for arrays should be equivalent.  The transformation of
the code however is quite drastic.  This transformation is similar to the
vectorisation transformation of NESL\cite{nesl}.  But since Accelerate provides
no way to automate it, the programmer needs to give up composability of
Accelerate terms.

This sharp division between scalar and array operations is easily the biggest
hindrance to the expressiveness of Accelerate. However, as this is a pervasive
part of the architecture of Accelerate, removing the distinction of
\texttt{Acc} and \texttt{Exp} would result in an entirely different language,
and every single backend would have to be rewritten.

The embedding of Accelerate leaves some things to be desired. It is for
instance not easily possible to pattern match on tuples and shapes, as these
need to be properly lifted and unlifted to be used (see above in function \texttt{g'}).

Also, lifting using \texttt{lift :: Lift c e => e -> c (Plain e)} uses the
\texttt{Plain} associated type, the definition of which is not shown in the
auto-generated documentation of instances. Although the documentation
generation system is arguably to blame for this, it does nonetheless make it
more difficult to easily use value lifting confidently.

There is no way to define recursive functions in Accelerate. Trying to do so
will result in compilation not terminating.

\section{Nikola}
\label{sec:language-discussion-nikola}

Nikola does not provide the variety of array operations that Accelerate does: Only
a few mapping operations are provided, and an iteration construct.  Nikola has
a lot of overall structure in common with Repa. As with Repa, an array's
implementations and shape are represented in the array type, and the programmer
has access to mutable arrays in addition to the common pure array operations.

To exploit the ability to differentiate according to array representations,
operations such as \texttt{map} and \texttt{zipWith} are implemented using
typeclasses.  While this architecture allows for specialisation of operations
to different array representations, actually using the operations results in quite
elaborate types. These are often too complicated for type inference to resolve
and require a human programmer to supply a type signature.

Compared to Accelerate, Nikola is embedded more naturally inside Haskell,
leveraging the \texttt{RebindableSyntax} GHC-extension. Also, in our programming
experience we were less required to interact with the value lifting machinery
than we were in Accelerate. Value lifting in Nikola is subject to the same
documentation inadequacies as we encountered with Accelerate.

While Nikola does not present any means for expressing nested parallelism, it
does not explicitly ban it either in the style of Accelerate's
\texttt{Acc}/\texttt{Exp} division. Thus, extending the expressive power of
Nikola in this aspect appears at first to be a less elaborate endeavour than in
Accelerate.

There is no way to define recursive functions in Nikola. Trying to do so
will result in compilation not terminating.

Nikola functions are compiled to CUDA code and then linked dynamically into
the running Haskell program. Then they are wrapped Haskell functions of
the corresponding type.  Exactly which types are possible is derived
from a menu of instances of the typeclass \texttt{Compilable a b}.

An instance of this typeclass denotes that a Nikola value of type \texttt{a}
may be compiled and wrapped into a regular Haskell value of type \texttt{b}.
There are instances for all the various Nikola values, and it is possible to
wrap Nikola arrays in many different vector implementations. Therefore it is
not in general the case that \texttt{a} determines \texttt{b}, and often one is
best served by just supplying a type signature.

However, this typeclass is not yet implemented for a lot of the
possible choices of array representations and combinations. During the
implementation of our case studies, we observed that trying to compile
a function with an instance missing often resulted in very lengthy and
elaborate type errors that were hard for us to decode. Luckily our
ability to decode compilation-related type errors improved over time,
but they are nevertheless still unpleasant.

Nikola also integrates with the array representation facility of Repa, by
implementing a representation for arrays that reside in the CUDA device's
memory. By using this array representation it is possible to have both Nikola
programs and other foreign functions manipulate the same physical memory withot
requiring any memory transfers of data to take place between host and device.

%%% Local Variables:
%%% mode: latex
%%% TeX-master: "../master"
%%% End:
