As concluded in our survey we want to experiment with extending Nikola in
pursuit of additional expressiveness. In order to give a more coherent picture
we therefore first give a short introduction of Nikola programming and the
parts of the implementation that we had to treat in order to make our
extensions.  Afterwards we present our takes on iterative array construction,
and we present a new method for directing data parallelism.

\vspace{1em}

Nikola is a deeply embedded, domain specific language in Haskell. It aims to
provide a means to better exploit hardware capabilities for parallel numerical
computation, as exemplified by OpenMP and (in particular) CUDA.

Surprisingly, we have found documentation on Nikola architecture and
programming scarce, with the Nikola article \cite{mainland2010nikola}
detailing mainly the embedding effort. In this chapter we try to make up for
this by describing the architecture of Nikola in sufficient detail to provide a
background for presenting our extensions.

First we present the frontend of Nikola, and then we describe the backend. Last
we devote a section to a small overview reference of Nikola modules.

