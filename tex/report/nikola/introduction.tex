\todo{The choice of nikola should already have been documented in the
conclusion of the survey.}

Nikola is a deeply embedded, domain specific language in haskell. It aims to
provide a means to better exploit hardware capabilities for parallel numerical
computation, as exemplified by OpenMP and (in particular) CUDA.

Surprisingly, we have found documentation on Nikola architecture and
programming scarce, with the Nikola article \cite{mainland2010nikola}
detailing mainly the embedding effort. In this chapter we try to make up for
this by describing the architecture of Nikola in sufficient detail to provide a
background for presenting our extensions.

First we present each of the components in detail and last we devote a section
to a small overview reference of Nikola modules.
