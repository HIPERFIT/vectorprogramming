\section{The Binomial Algorithm}

A relatively simple model for computing the price of an american style option
is the \emph{standard binomial model} \todo{(citation needed) (mentioned by
Rolf).}. The finance background for this model is beyond the
scope of this thesis.

Operationally, the model works by estimating the price of the option in lattice
points. The price estimate at each lattice point for time $t$ depends only on
the estimates of the two neighboring points at time $t+1$. Thus, the dependency
graph has the shape of a binomial tree. 

\todo{Nice picture here, like the one in Rolfs FAMØS article.}

Since the value at expiration time $T$ is given by the contract parameters as
the difference between the strike price $k$ and price of the underlying asset,
prices are successively estimated from $t=T$ to $t=0$.

\todo{Display the full binomial algorithm and parameters in a concise fact-box
like format. Be sure to explain what $S(i,t)$ is.}

In equations, this may be written as follows, where $x_{i,t}$ represents a
lattice point.

\begin{align*}
v(x_{i,t}) &= h(f_{i,t}(v(x_{i,t+1})),g_{i,t}(v(x_{i+1,t+1}))) \\
v(x_{i,T}) &= \max\{ 0, k - S(i,T) \} \tag{for a put-option}\\
v(x_{i,T}) &= \max\{ 0, S(i,T) - k \} \tag{for a call-option}\\
\end{align*}

In our case the functions $f_{i,t},g_{i,t},h$ contain only simple constant-time
arithmetic calculation.

\subsection{Implmentation Techniques}

As seen above, the value of $v(x_{i,t})$ depends only on values
associated with time $t+1$. Therefore, for a given $t$, the evaluation of
$v(x_{i,t})$ for each desired value of $i$ may proceed in parallel.

Below, this is illustrated in pseudocode that uses only two arrays in tandem
to hold the intermediately produced values.

\todo{format nicely as pseudocode, possibly given a number to refer to in the survey later}
\begin{verbatim}
binom:
  -- A,B are assumed preallocated, and T is assumed odd.
  t <- T
  B <- parmap (\i -> max 0 (k-S(i,T))) [0..T]
  while t > 0
    A <- parmap (\i -> h(f_{i,t}(B[i]),g_{i,t}(B[i+1))) [0..t]
    t <- t -1
    B <- parmap (\i -> h(f_{i,t}(A[i]),g_{i,t}(A[i+1))) [0..t]
    t <- t -1
  return B
\end{verbatim}

\todo{cite blelloch} As defined by Blelloch (citation needed), the above code
may be reasonably assumed to have time complexities $Work(T^2/2)$ and
$Depth(T)$, and memory complexity $O(2T)$.

While the above code is both short and simple in structure, the main motive for
pursuing parallelism is in gaining performance rather than expressivity.  To
explore further algorithm samples, we examine the binomial pricing algorithm
included in the CUDA developer SDK.\todo{cite cuda binom documentation}

\todo{format nicely as pseudocode, possibly given a number to refer to in the survey later}
\begin{verbatim}
lots of pseudocode.
\end{verbatim}

Notes: Duplication of computation, fixed amount of parallelism pr option being
priced, better scaling to pricing a larger number of options. Remark on the
poor scaling of the first algoritm.
