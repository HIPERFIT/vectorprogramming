\section{Binomial option pricing}

\subsection{Options}

Finance as a mathematical discipline is concerned with the pricing of assets.
Financial assets can be material objects such as goods, or contracts on other
assets. The \emph{stock} of a company is a common example of a financial contract.

However, as this definition of a financial asset is recursive, contracts on
other contracts are possible. An
\emph{option} to buy or sell some other asset for some price at some point in
time is a common example, and the one that we will invest some of our effort
treating. These types of contracts are commonly termed \emph{financial derivatives}.


\emph{Option} is the name used collectively for a range of different financial
contracts that all share some features, namely that an option on an asset $A$
gives the holder the right to exchange a certain amount of $A$'s for a certain
strike price $k$ in a certain duration of time up until expiration time $T$.

The different types of options vary then in which exact processes determine the
size of strike price and the allowed time span where the option may be
exercised.

We make a distinction between options that grants a right to buy and those that
grants a right to sell. A \emph{call option} on asset $A$ grants the right to
buy $A$'s, whereas a \emph{put option} grants the right to sell.

For a concrete example, consider this:
\begin{example}[European Call Option]

  European options\footnote{The continent of Europe is completely unrelated to
  the naming of this contract.} are characterised by having a pre-determined
  constant strike price $k$, and may be exercised only on the expiration time
  $T$.

  So, having aquired a european call option for 1 of IBM company stock with
  strike price \$100 and expiration time January 1st, we wait.  When upon the
  arrival of January 1st we observe that IBM stock is now exchanged valued
  \$110, we choose to exercise our option right thus recieve 1 stock for \$100.
  Immediately, we sell this stock on the market for \$110 and have thus earned
  \$10. Our option was ``in the money ''.

  Note that in reality, the underlying asset rarely changes hands. Instead,
  only the net difference in strike price and market price is
  exchanged.\footnote{A consequence of this is that you are not required to
  actually own any quantity of the underlying asset for which you issue an
  option for.}

\end{example}

The challenge of option contracts resides in how to determine their price. The
complexity of that challenge varies along with the differences in parameters.

The value of european options for instance is closely approximated by the Black
Scholes formula (citation needed), which is a closed form analytical solution.

In contrast to european options lie american options\footnote{which are
similarly irrelevantly named}, characterised by granting the right to exercise
at any point in time up until expiration, rather than only at expiration.

This difference in granted rights makes amercan options require a different
pricing method than european options. But as of today, only nummerical
approximation schemes are known for this type of contract. While this may be
detrimental for those that depend on accurate pricing of their financial
assets, it's all the more interesting for researchers in computating to treat.

\subsection{The Binomial Algorithm}

%One of the simplest algorithms for pricing American options is based
%on a binomial probability model.
%Using this method, we assume that the
%value of the underlying asset moves a fixed percentage up or down at
%each time step. We usually assume that the probability of going up at
%each time step corresponds to the riskless rate.

We can then represent all possible states from the current time to the
last possible exercise time as binomial tree of up and down
movements. The root carries the current value, its two children
carries the value after one up or down movement, and so forth. The
leafs then represents the value at the last time step.

%%% Local Variables:
%%% mode: latex
%%% TeX-master: "master"
%%% End:
