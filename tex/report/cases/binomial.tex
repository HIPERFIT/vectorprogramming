\section{The Binomial Algorithm}

A relatively simple model for computing the price of an american style option
is the \emph{standard binomial model} \todo{(citation needed) (mentioned by
Rolf).}. The finance background for this model is beyond the
scope of this thesis.

Operationally, the model works by estimating the price of the option in
lattice points, connected to form a binomial tree.

\todo{Nice picture here, like the one in Rolfs FAMØS article.}

The edge between lattice

%One of the simplest algorithms for pricing American options is based
%on a binomial probability model.
%Using this method, we assume that the
%value of the underlying asset moves a fixed percentage up or down at
%each time step. We usually assume that the probability of going up at
%each time step corresponds to the riskless rate.

We can then represent all possible states from the current time to the
last possible exercise time as binomial tree of up and down
movements. The root carries the current value, its two children
carries the value after one up or down movement, and so forth. The
leafs then represents the value at the last time step.
