\section{Option pricing}
Finance as a mathematical discipline is concerned with the pricing of
assets.  Financial assets can be material objects such as goods, or
contracts on other assets. The \emph{stock} issued by a company is a
common example of a financial contract. However, as this definition of
a financial asset is recursive, contracts on other contracts are
possible. An \emph{option} to buy or sell some other asset for some
price at some point in time is a common example, and the one that we
will invest some of our effort treating. These types of contracts are
commonly termed \emph{financial derivatives}.

\emph{Option} is the name used collectively for a range of different
financial contracts that all share some features, namely that an
option on an underlying asset $A$ gives the holder the right to
exchange $A$ for a certain strike price $K$ in a certain time interval
up until a prespecified expiration time $T$.

The different types of options vary the processes that determine the
size of strike price and the time span where the option may be
exercised.  We make a distinction between options that grants a right
to buy and those that grants a right to sell. A \emph{call option} on
asset $A$ grants the right to buy $A$, whereas a \emph{put option}
grants the right to sell.

For a concrete example, consider this:
\begin{example}[European Call Option]
  European options\footnote{The continent of Europe is completely unrelated to
  the naming of this class of contracts.} are characterised by having a pre-determined
  constant strike price $K$, and may be exercised only on the expiration time
  $T$.

  \begin{itemize}
  \item Assume that we at time $t_0$ acquire an European call option
    for the stock $A$ which is quoted as $S(t_0)=\$90$, and that the
    option has a strike price of $K=\$100$ and expiration time
    $T=t_1$.

    % \item So, having acquired a European call option for 1 of IBM company stock
    %   with strike price \$100 and expiration time January 1st, we wait.
    %   \todo{What is $S_0$ in this example?}

  \item When upon the arrival of $t_1$ we observe that $A$ is now
    exchanged at value $S(t_1)=\$105$, we choose to exercise our option right
    and thus recieve one $A$ for \$100.

    \item Immediately, we sell this stock on the market for \$105 and have thus
      earned $S(t_1)-K=\$5$. Our option was ``in the money''.
  \end{itemize}

  Note that in reality, the underlying asset rarely changes
  hands. Instead, only the net difference in strike price and market
  price is exchanged.\footnote{A consequence of this is that you are
    not required to actually own any quantity of the underlying asset
    for which you issue an option for.}
\end{example}

The challenge of option contracts resides in how to determine their
price. The above option would have been appropriately priced at $\$5$
discounted to $t_0$ prices. The complexity of the pricing challenge
varies along with the differences in parameters.  The value of
European options for instance is closely approximated by the
Black-Scholes formula \cite{black1973pricing}, which is a closed form
analytical solution.  In contrast to European options lie American
options\footnote{Which are similarly irrelevantly named.},
characterised by granting the right to exercise at any point in time
up until expiration, rather than only at expiration.

This difference in granted rights makes American options require a different
pricing method than European options. But as of today, only numerical
approximation schemes are known for this type of contract. While this may be
detrimental for those that depend on accurate pricing of their financial
assets, it is all the more interesting for researchers of computing to treat.

There are three classes of algorithms for American options, finite
difference, lattice and Monte Carlo based methods
\cite{ibanez2004monte}. The ensuing sections will introduce examples
of both lattice and Monte Carlo based methods. We will not discuss
finite difference methods.

%%% Local Variables:
%%% mode: latex
%%% TeX-master: "../master"
%%% End:
