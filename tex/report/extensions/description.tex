\section{Frontend Architecture}

In this section we describe the part of Nikola that serves as the language
programming interface, as well as the middle layer which is still backend
agnostic.

The Nikola language is made up of three data types: \lstinline{Exp t a},
\lstinline{Array r sh a} and the program monad \lstinline{P a}. These parts are all
backend agnostic. It also provides a type class directed \lstinline{compile}
function for compiling Nikola functions to CUDA code which is subsequently
wrapped as regular haskell functions. \lstinline{Exp t a} expressions are first
translated to another first-order untyped abstract syntax before CUDA code
generation.

\paragraph{Nikola scalar expressions} are represented by the \lstinline{Exp t a} type, with
the phantom type variable \lstinline{t} representing the target architecture for
the expression, eg.  \lstinline{CUDA}. Thus, it is possible to have Nikola terms
specialised for different backends. This is unusual, as programming languages
are typically assumed universal in the sense that every well-typed expression
is executable on every supported machine architecture. Nikola is the first
programming language we have witnessed to explicitly encode term portability in
the type system. However, the ability to specialise expressions to targets
doesn't extend into the lower layers of Nikola, so as such the \lstinline{t} type
variable only clutters up type signatures currently.

\paragraph{Arrays in Nikola} are modeled by the type \lstinline{Array r sh a}, an associated
type of the typeclass \lstinline{IsArray r a}. Arrays are parametrised on their
representation and shape by the type variables \lstinline{r} and \lstinline{sh},
following in the tradition of Repa. The only common operation for all arrays is
that of extracting their shape through the function \lstinline{extent}. Array
shapes are similar to those of Repa and Accelerate, denoted as instances of
typeclass \lstinline{Shape sh}.

Three principal array representations are provided by Nikola: The global array,
the delayed array, and the push array, denoted respectably by the types
\lstinline{G}, \lstinline{PSH} and \lstinline{D}. This is a source of both control and
complexity, as each representation gives rise to different features and
restrictions. The global array represents a certain manifest range of memory
cells, from which it is only referentially safe to read. Push arrays and
delayed arrays however, represent array computations rather than actual areas
of memory - only at the Nikola-Haskell border are they manifest into memory. An
important consequence of this is that terms composed of delayed and push arrays
undergo fusion by construction.

What distinguishs push arrays and delayed arrays is their perspective on the
arrays they represent. A delayed array is simply a function from indices to
values, while a push array may be viewed as a stream of index/value pairs, that
may appear in potentially any ordering.

\paragraph{The type of the low-level abstract syntax} is \lstinline{S.Exp},
qualified to avoid confusion with \lstinline{Exp t a}. This is elaborated in the
nikola reference in section \ref{section:nikola-reference}.  \lstinline{S.Exp}
defines primitive constructs for annonymous functions, delayed arrays,
accessing and manipulation of mutable arrays, and a \lstinline{for}-loop
construct. Many of these constructs map directly to corresponding C constructs.

\paragraph{The monad \lstinline{P a}} serves some of the same roles as the
\lstinline{IO} monad does in plain Haskell. One must for example only manipulate
mutable arrays from within the \lstinline{P} monad.  It is a type alias for the
slightly more elaborate monad type \lstinline{type P a = R S.Exp a}. The monad
\lstinline{R r a} is the reification monad, used to convert the various
frontend datatypes such as \lstinline{Exp t a} and \lstinline{Array r sh a} into the
low-level abstract syntax \lstinline{S.Exp}.

To enable this, monad \lstinline{R r a} is a continuation monad.  % \todo{for
% now just assume knowledge of continuation passing monads :-(}
  But instead of direct access to the underlying continuation, Nikola
  uses delimited continuations, described in \cite{wadler1994monads}
  and introduced first in \cite{filinski1996controlling}.  The
  interface to using delimited continuations consists of two special
  operations:

\begin{lstlisting}
shift :: ((a -> R r r) -> R r r) -> R r a
reset :: R r a -> R r r
\end{lstlisting}

Which when specialised to the \lstinline{P} monad becomes:

\begin{lstlisting}
shift :: ((a -> P S.Exp) -> P S.Exp) -> P a
reset :: P a -> P S.Exp
\end{lstlisting}

\lstinline{shift} and \lstinline{reset} work in tandem. A full account of the use and
details of delimited continuations is out of scope for this thesis, but
consider this short typical usage pattern:

\begin{lstlisting}
reset $ do
  ...
  y <- shift $ \k -> do
    ...
    x <- k ""
    ...
  ...

\end{lstlisting}

In this \lstinline{shift} expression, \lstinline{k} is a continuation representing
all that is going to happen up until the enclosing \lstinline{reset}.  Upon
invoking \lstinline{k ""}, control shifts outside of \lstinline{shift}, and \lstinline{y}
is bound to the empty string \lstinline{""}. Upon reaching the end of the monadic
action inside \lstinline{reset}, control is shifted back into \lstinline{shift}, and
\lstinline{x} is bound to the result of the enclosing action. The eventual result
of the action inside \lstinline{shift} then becomes the result of the enclosing
\lstinline{reset}. Informally, \lstinline{shift} and \lstinline{reset} turn the code
inside-out.

\section{Nikola Backend Architecture}

Compiling a Nikola function relies on both reification to \lstinline{S.Exp}, a
mechanism to determine the type of the resulting Haskell function, and the
translation from \lstinline{S.Exp} abstract syntax to C abstract syntax.
Reification is handled by the typeclass \lstinline{Reifiable a b}, and the
Haskell-interfacing parts of compilation by \lstinline{Compilable a b}.  The
details of how Nikola manages to interface Haskell with the compiled Nikola
code is documented in \cite{mainland2010nikola}.

All it takes for a construct to be part of the Nikola language frontend is
proper instances of these two type classes. So as such, Nikola is very
extensible, provided that the programmer is capable of lifting the burden of
expressing his additions in terms of the low-level abstract syntax.

Once a term has undergone reification, it is translated into C abstract syntax.
This is a somewhat straightforward translation, as many of the components
correspond directly to C.

%%% Local Variables: 
%%% mode: latex
%%% TeX-master: "../master"
%%% End: 
