\chapter{Nikola}
As concluded in our survey we want to experiment with extending Nikola in
pursuit of additional expressiveness. In order to give a more coherent picture
we therefore first give a short introduction of Nikola programming and the
parts of the implementation that we had to treat in order to make our
extensions.  Afterwards we present our takes on iterative array construction,
and we present a new method for directing data parallelism.

\vspace{1em}

Nikola is a deeply embedded, domain specific language in Haskell. It aims to
provide a means to better exploit hardware capabilities for parallel numerical
computation, as exemplified by OpenMP and (in particular) CUDA.

Surprisingly, we have found documentation on Nikola architecture and
programming scarce, with the Nikola article \cite{mainland2010nikola}
detailing mainly the embedding effort. In this chapter we try to make up for
this by describing the architecture of Nikola in sufficient detail to provide a
background for presenting our extensions.

First we present the frontend of Nikola, and then we describe the backend. Last
we devote a section to a small overview reference of Nikola modules.

\section{Nikola Architecture}
\todo{Describe the various parts of the Nikola
architecture, i.e. the role of S.Exp, the various monads, how interfacing with
nikola programs using various vector implementations is supposed to work, what
substantial parts are missing...}

\section{Nikola module overiew reference} \label{section:nikola-reference}

In this section we provide a small overview of central modules and namespaces
in Nikola for reference. These modules all reside in the
\texttt{Data.Array.Nikola} namespace.

\subsection{Frontend modules}
\begin{description}

  \item[\texttt{Exp}] module. Exports the type \texttt{Exp t a} and typeclass
    \texttt{IfThenElse} for use with rebindable syntax. The type \texttt{Exp t
    a} will sometimes appear as qualified name \texttt{E.Exp} in case of
    ambiguity.

  \item[\texttt{Array}] module. Defines \texttt{IsArray r e} typeclass with
    associated type \texttt{Array r sh e}. It also defines the
    \texttt{Source r e} typeclass for arrays that support indexing.

  \item[\texttt{Shape}] module. Defines types \texttt{Z}, \texttt{sh :. a}
    and the typeclass \texttt{Shape sh}, with instances for \texttt{Z} and
    \texttt{sh :. a}.

  \item[\texttt{Repr}] namespace. Contains modules definining the array
    representations for Push, Delayed and Global arrays through
    \texttt{IsArray} instances.

  \item[\texttt{Backend.CUDA.Haskell}, \texttt{Backend.CUDA.TH}] modules.
    Define the \texttt{compile} function for runtime and compiletime usage
    respectively.

\end{description}

\subsection*{Backend modules}
\begin{description}

  \item[\texttt{Language.Syntax}] module. Exports the \texttt{Exp} datatype,
    which is a first-order abstract syntax for primitive Nikola expressions.
    The abstract syntax contains constructs for manipulating memory arrays,
    looping and annonymous functions. It also exports the datatypes
    \texttt{Type} and \texttt{ScalarType} used as church style type annotations
    for \texttt{Exp} values. In the case of ambiguity we quailfy \texttt{Exp}
    as \texttt{S.Exp}.

  \item[\texttt{Language.Check}] module. Contains a type checker and simple type
    inference for \texttt{Exp} values. Also defines the typeclass
    \texttt{MonadCheck m}, to allow type checking to be performed in any monad
    that defines how to access to the typing environment

  \item[\texttt{Language.Monad}] module. Defines the continuation passing
    \texttt{R r a} monad, and the type alias \texttt{type P a = R Exp a}.
    Also defines the combinators \texttt{shift} and \texttt{reset} for
    delimited continuations \cite{waddler1994monads}.

  \item[\texttt{Language.Reify}] module. Exports the \texttt{Reifiable a b}
    typeclass. It provides the operation \texttt{reify :: a -> R b b},
    which is the central piece in converting frontend data types
    \texttt{Exp t a} and \texttt{Array r sh e} to primitive \texttt{Exp}.

  \item[\texttt{Language.Optimize}] namespace and module. Provides various
    optimizations on Nikola \texttt{S.Exp} values exclusively. Most notable is
    the detection of let-sharing, see \cite{mainland2010nikola}. Marking
    parallel for-loops at the top level to be translated to CUDA kernels is
    also treated as an optimisation in Nikola.

  \item[\texttt{Backend.C.Codegen}] module. Generates C code from \texttt{S.Exp}
    values, represented as C abstract syntax.

\end{description}

Apart from these modules Nikola also provides various implementations of
vectors in the \texttt{Data.Vector.CUDA} namespace. These vectors are
stored in device memory and integrate with \texttt{compile} to allow
control of the timing of host-device data transfers.

Furthermore these vector implementations are integrated with Repa as additional
array representation types in modules in the \texttt{Data.Array.Repa.Repr.CUDA}
namespace.

\subsection{Array representation capabilities.}

The operations that arrays may be used with is modeled with the use of type
classes.

\begin{itemize}

  \item The \texttt{Source r e} type class specifies an \texttt{index}
    function that allow a consumer to extract values of type \texttt{e} from an
    array with representation \texttt{r} and contents \texttt{e}.

    Instances:
    \begin{itemize}
      \item \texttt{Source D e}
      \item \texttt{IsElem e => Source G e}
    \end{itemize}

  \item The \texttt{Target r e} type class enables mutable access of
    \texttt{r}-arrays in the \texttt{P} monad.

    Instances:
    \begin{itemize}
      \item \texttt{IsElem e => Target G e}
    \end{itemize}

  \item The \texttt{Load r sh e} type class enables manifestation of
    \texttt{r}-arrays into \texttt{Target r' e} arrays in the \texttt{P} monad.

    Instances:
    \begin{itemize}
      \item \texttt{Shape sh => Load sh D}
      \item \texttt{Shape sh => Load sh PSH}
    \end{itemize}
\end{itemize}
