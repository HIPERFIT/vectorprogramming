\chapter{Nikola}
As concluded in our survey we want to experiment with extending Nikola in
pursuit of additional expressiveness. In order to give a more coherent picture
we therefore first give a short introduction of Nikola programming and the
parts of the implementation that we had to treat in order to make our
extensions.  Afterwards we present our takes on iterative array construction,
and we present a new method for directing data parallelism.

\vspace{1em}

Nikola is a deeply embedded, domain specific language in Haskell. It aims to
provide a means to better exploit hardware capabilities for parallel numerical
computation, as exemplified by OpenMP and (in particular) CUDA.

Surprisingly, we have found documentation on Nikola architecture and
programming scarce, with the Nikola article \cite{mainland2010nikola}
detailing mainly the embedding effort. In this chapter we try to make up for
this by describing the architecture of Nikola in sufficient detail to provide a
background for presenting our extensions.

First we present the frontend of Nikola, and then we describe the backend. Last
we devote a section to a small overview reference of Nikola modules.

\section{Frontend Architecture}

In this section we describe the part of Nikola that serves as the language
programming interface, as well as the middle layer which is still backend
agnostic.

The Nikola language is made up of three data types: \texttt{Exp t a},
\texttt{Array r sh a} and the program monad \texttt{P a}. These parts are all
backend agnostic. It also provides a type class directed \texttt{compile}
function for compiling Nikola functions to CUDA code which is subsequently
wrapped as regular haskell functions. \texttt{Exp t a} expressions are first
translated to another first-order untyped abstract syntax before CUDA code
generation.

\paragraph{Nikola scalar expressions} are represented by the \texttt{Exp t a} type, with
the phantom type variable \texttt{t} representing the target architecture for
the expression, eg.  \texttt{CUDA}. Thus, it is possible to have Nikola terms
specialised for different backends. This is unusual, as programming languages
are typically assumed universal in the sense that every well-typed expression
is executable on every supported machine architecture. Nikola is the first
programming language we have witnessed to explicitly encode term portability in
the type system. However, the ability to specialise expressions to targets
doesn't extend into the lower layers of Nikola, so as such the \texttt{t} type
variable only clutters up type signatures currently.

\paragraph{Arrays in Nikola} are modeled by the type \texttt{Array r sh a}, an associated
type of the typeclass \texttt{IsArray r a}. Arrays are parametrised on their
representation and shape by the type variables \texttt{r} and \texttt{sh},
following in the tradition of Repa. The only common operation for all arrays is
that of extracting their shape through the function \texttt{extent}. Array
shapes are similar to those of Repa and Accelerate, denoted as instances of
typeclass \texttt{Shape sh}.

Three principal array representations are provided by Nikola: The global array,
the delayed array, and the push array, denoted respectably by the types
\texttt{G}, \texttt{PSH} and \texttt{D}. This is a source of both control and
complexity, as each representation gives rise to different features and
restrictions. The global array represents a certain manifest range of memory
cells, from which it is only referentially safe to read. Push arrays and
delayed arrays however, represent array computations rather than actual areas
of memory - only at the Nikola-Haskell border are they manifest into memory. An
important consequence of this is that terms composed of delayed and push arrays
undergo fusion by construction.

What distinguishs push arrays and delayed arrays is their perspective on the
arrays they represent. A delayed array is simply a function from indices to
values, while a push array may be viewed as a stream of index/value pairs, that
may appear in potentially any ordering.

\paragraph{The type of the low-level abstract syntax} is \texttt{S.Exp},
qualified to avoid confusion with \texttt{Exp t a}. This is elaborated in the
nikola reference in section \ref{section:nikola-reference}.  \texttt{S.Exp}
defines primitive constructs for annonymous functions, delayed arrays,
accessing and manipulation of mutable arrays, and a \texttt{for}-loop
construct. Many of these constructs map directly to corresponding C constructs.

\paragraph{The monad \texttt{P a}} serves some of the same roles as the
\texttt{IO} monad does in plain Haskell. One must for example only manipulate
mutable arrays from within the \texttt{P} monad.  It is a type alias for the
slightly more elaborate monad type \texttt{type P a = R S.Exp a}. The monad
\texttt{R r a} is the reification monad, used to convert the various
frontend datatypes such as \texttt{Exp t a} and \texttt{Array r sh a} into the
low-level abstract syntax \texttt{S.Exp}.

To enable this, monad \texttt{R r a} is a continuation monad.  \todo{for
now just assume knowledge of continuation passing monads :-(} But instead of
direct access to the underlying continuation, Nikola uses delimited
continuations, described in \cite{wadler1994monads} and introduced first in
\cite{filinski1996controlling}.
The interface to using delimited continuations consists of two special operations:

\begin{verbatim}
shift :: ((a -> R r r) -> R r r) -> R r a
reset :: R r a -> R r r
\end{verbatim}

Which when specialised to the \texttt{P} monad becomes:

\begin{verbatim}
shift :: ((a -> P S.Exp) -> P S.Exp) -> P a
reset :: P a -> P S.Exp
\end{verbatim}

\texttt{shift} and \texttt{reset} work in tandem. A full account of the use and
details of delimited continuations is out of scope for this thesis, but
consider this short typical usage pattern:

\begin{verbatim}
reset $ do
  ...
  y <- shift $ \k -> do
    ...
    x <- k ""
    ...
  ...

\end{verbatim}

In this \texttt{shift} expression, \texttt{k} is a continuation representing
all that is going to happen up until the enclosing \texttt{reset}.  Upon
invoking \texttt{k ""}, control shifts outside of \texttt{shift}, and \texttt{y}
is bound to the empty string \texttt{""}. Upon reaching the end of the monadic
action inside \texttt{reset}, control is shifted back into \texttt{shift}, and
\texttt{x} is bound to the result of the enclosing action. The eventual result
of the action inside \texttt{shift} then becomes the result of the enclosing
\texttt{reset}. Informally, \texttt{shift} and \texttt{reset} turn the code
inside-out.

\section{Nikola Backend Architecture}

In this section we describe some parts of Nikola that are used for code
generation.

Compiling a Nikola function relies on both reification to \texttt{S.Exp}, a
mechanism to determine the type of the resulting haskell function, and the
translation from \texttt{S.Exp} abstract syntax to C abstract syntax.

Reification is handled by the function \texttt{reify :: Reifiable a b => a -> R
b b} of typeclass \texttt{Reifiable}.

The programmer is \ldots \todo{explain how Compilable a b instance selection allows for different resulting haskell wrapper types}

\section{Nikola module overiew reference} \label{section:nikola-reference}

In this section we provide a small overview of central modules and namespaces
in Nikola for reference. These modules all reside in the
\texttt{Data.Array.Nikola} namespace.

\subsection{Frontend modules}
\begin{description}

  \item[\texttt{Exp}] module. Exports the type \texttt{Exp t a} and typeclass
    \texttt{IfThenElse} for use with rebindable syntax. The type \texttt{Exp t
    a} will sometimes appear as qualified name \texttt{E.Exp} in case of
    ambiguity.

  \item[\texttt{Array}] module. Defines \texttt{IsArray r e} typeclass with
    associated type \texttt{Array r sh e}. It also defines the
    \texttt{Source r e} typeclass for arrays that support indexing.

  \item[\texttt{Shape}] module. Defines types \texttt{Z}, \texttt{sh :. a}
    and the typeclass \texttt{Shape sh}, with instances for \texttt{Z} and
    \texttt{sh :. a}.

  \item[\texttt{Repr}] namespace. Contains modules definining the array
    representations for Push, Delayed and Global arrays through
    \texttt{IsArray} instances.

  \item[\texttt{Backend.CUDA.Haskell}, \texttt{Backend.CUDA.TH}] modules.
    Define the \texttt{compile} function for runtime and compiletime usage
    respectively.

\end{description}

\subsection*{Backend modules}
\begin{description}

  \item[\texttt{Language.Syntax}] module. Exports the \texttt{Exp} datatype,
    which is a first-order abstract syntax for primitive Nikola expressions.
    The abstract syntax contains constructs for manipulating memory arrays,
    looping and annonymous functions. It also exports the datatypes
    \texttt{Type} and \texttt{ScalarType} used as church style type annotations
    for \texttt{Exp} values. In the case of ambiguity we quailfy \texttt{Exp}
    as \texttt{S.Exp}.

  \item[\texttt{Language.Check}] module. Contains a type checker and simple type
    inference for \texttt{Exp} values. Also defines the typeclass
    \texttt{MonadCheck m}, to allow type checking to be performed in any monad
    that defines how to access to the typing environment

  \item[\texttt{Language.Monad}] module. Defines the continuation passing
    \texttt{R r a} monad, and the type alias \texttt{type P a = R Exp a}.
    Also defines the combinators \texttt{shift} and \texttt{reset} for
    delimited continuations \cite{wadler1994monads}.

  \item[\texttt{Language.Reify}] module. Exports the \texttt{Reifiable a b}
    typeclass. It provides the operation \texttt{reify :: a -> R b b},
    which is the central piece in converting frontend data types
    \texttt{Exp t a} and \texttt{Array r sh e} to primitive \texttt{Exp}.

  \item[\texttt{Language.Optimize}] namespace and module. Provides various
    optimizations on Nikola \texttt{S.Exp} values exclusively. Most notable is
    the detection of let-sharing, see \cite{mainland2010nikola}. Marking
    parallel for-loops at the top level to be translated to CUDA kernels is
    also treated as an optimisation in Nikola.

  \item[\texttt{Backend.C.Codegen}] module. Generates C code from \texttt{S.Exp}
    values, represented as C abstract syntax.

\end{description}

Apart from these modules Nikola also provides various implementations of
vectors in the \texttt{Data.Vector.CUDA} namespace. These vectors are
stored in device memory and integrate with \texttt{compile} to allow
control of the timing of host-device data transfers.

Furthermore these vector implementations are integrated with Repa as additional
array representation types in modules in the \texttt{Data.Array.Repa.Repr.CUDA}
namespace.

\subsection{Array representation capabilities.}

The operations that arrays may be used with is modeled with the use of type
classes.

\begin{itemize}

  \item The \texttt{Source r e} type class specifies an \texttt{index}
    function that allow a consumer to extract values of type \texttt{e} from an
    array with representation \texttt{r} and contents \texttt{e}.

    Instances:
    \begin{itemize}
      \item \texttt{Source D e}
      \item \texttt{IsElem e => Source G e}
    \end{itemize}

  \item The \texttt{Target r e} type class enables mutable access of
    \texttt{r}-arrays in the \texttt{P} monad.

    Instances:
    \begin{itemize}
      \item \texttt{IsElem e => Target G e}
    \end{itemize}

  \item The \texttt{Load r sh e} type class enables manifestation of
    \texttt{r}-arrays into \texttt{Target r' e} arrays in the \texttt{P} monad.

    Instances:
    \begin{itemize}
      \item \texttt{Shape sh => Load sh D}
      \item \texttt{Shape sh => Load sh PSH}
    \end{itemize}
\end{itemize}
