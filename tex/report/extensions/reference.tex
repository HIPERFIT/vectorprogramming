\section{Nikola module overiew reference} \label{section:nikola-reference}

In this section we provide a small overview of central modules and namespaces
in Nikola for reference. These modules all reside in the
\lstinline{Data.Array.Nikola} namespace.

\subsection{Exposed frontend modules}
\begin{description}

\item[\texttt{Exp}] Exports the type \lstinline{Exp t a} and typeclass
  \lstinline{IfThenElse} for use with rebindable syntax. The type
  \lstinline{Exp t a} will sometimes appear as qualified name
  \lstinline{E.Exp} in case of ambiguity.

  \item[\texttt{Array}] Defines \lstinline{IsArray r e} typeclass with
    associated type \lstinline{Array r sh e}. It also defines the
    \lstinline{Source r e} typeclass for arrays that support indexing.

  \item[\texttt{Shape}] Defines types \lstinline{Z}, \lstinline{sh :. a}
    and the typeclass \lstinline{Shape sh}, with instances for \lstinline{Z} and
    \lstinline{sh :. a}.

  \item[\texttt{Repr.*}] Contains modules definining the array
    representations for Push, Delayed and Global arrays through
    \lstinline{IsArray} instances.

\end{description}

\subsection*{Low-level frontend modules}
\begin{description}

  \item[\texttt{Language.Syntax}] Exports the \lstinline{Exp} datatype,
    which is a first-order abstract syntax for primitive Nikola expressions.
    The abstract syntax contains constructs for manipulating memory arrays,
    looping and annonymous functions. It also exports the datatypes
    \lstinline{Type} and \lstinline{ScalarType} used as church style type annotations
    for \lstinline{Exp} values. In the case of ambiguity we quailfy \lstinline{Exp}
    as \lstinline{S.Exp}.

  \item[\texttt{Language.Check}] Contains a type checker and simple type
    inference for \lstinline{Exp} values. Also defines the typeclass
    \lstinline{MonadCheck m}, to allow type checking to be performed in any monad
    that defines how to access to the typing environment

  \item[\texttt{Language.Monad}] Defines the continuation passing
    \lstinline{R r a} monad, and the type alias \lstinline{type P a = R Exp a}.
    Also defines the combinators \lstinline{shift} and \lstinline{reset} for
    delimited continuations \cite{wadler1994monads}.

  \item[\texttt{Language.Reify}] Exports the \lstinline{Reifiable a b}
    typeclass. It provides the operation \lstinline{reify :: a -> R b b},
    which is the central piece in converting frontend data types
    \lstinline{Exp t a} and \lstinline{Array r sh e} to primitive \lstinline{Exp}.

  \item[\texttt{Language.Optimize.*}] Provides various
    optimizations on Nikola \lstinline{S.Exp} values exclusively. Most notable is
    the detection of let-sharing, see \cite{mainland2010nikola}. Marking
    parallel for-loops at the top level to be translated to CUDA kernels is
    also treated as an optimisation in Nikola.

\end{description}

\subsection*{Backend modules}

\begin{description}

  \item[\texttt{Backend.CUDA.\{Haskell,TH\}}]
    Define the \lstinline{compile} function for runtime and compiletime usage
    respectively. This is the exposed part of the backend.

  \item[\texttt{Backend.C.Codegen}] Generates C code from
    \lstinline{S.Exp} values, represented as C abstract syntax.

\end{description}

Apart from these modules Nikola also provides various implementations of
vectors in the \lstinline{Data.Vector.CUDA} namespace. These vectors are
stored in device memory and integrate with \lstinline{compile} to allow
control of the timing of host-device data transfers.

Furthermore these vector implementations are integrated with Repa as additional
array representation types in modules in the \lstinline{Data.Array.Repa.Repr.CUDA}
namespace.

\subsection{Array representation capabilities.}
The operations that arrays may be used with is modeled with the use of type
classes.

\begin{itemize}
  \item The \lstinline{Source r e} type class specifies an \lstinline{index}
    function that allow a consumer to extract values of type \lstinline{e} from an
    array with representation \lstinline{r} and contents \lstinline{e}.

    Instances:
\begin{lstlisting}
Source D e
IsElem e => Source G e
\end{lstlisting}
  \item The \lstinline{Target r e} type class enables mutable access of
    \lstinline{r}-arrays in the \lstinline{P} monad.

    Instances:
\begin{lstlisting}
IsElem e => Target G e
\end{lstlisting}
  \item The \lstinline{Load r sh e} type class enables manifestation of
    \lstinline{r}-arrays into \lstinline{Target r' e} arrays in the \lstinline{P} monad.

    Instances:
\begin{lstlisting}
Shape sh => Load sh D 
Shape sh => Load sh PSH
\end{lstlisting}

\end{itemize}

%%% Local Variables: 
%%% mode: latex
%%% TeX-master: "../master"
%%% End: 
