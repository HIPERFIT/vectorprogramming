\chapter{Directing parallelism}
\label{chap:directing-parallelism}
% * Motivation
%   - NDP
%   - Search for alternatives to full vectorisation
%   - Vectorisation overhead

% * What we have done and our progress with implementation

% * Presentation of sequential
%   - A few small examples
%   - Use-cases in our option pricing examples

% * Discussion
%   - Presentation of alternatives
%      * Repa computeS/computeP
%      * Type tagging

%   - Interactions with fusion

%   - How it allows you to write programs much in the same style as
%     you currently are writing CUDA programs.

%   - How it will introduce possibilities for runtime errors (by
%     irregularity), and that we do not have a proposal for how to solve
%     this, and we are not that happy about that aspect.

%   - We believe that we can obtain comparable performance to code
%     written directly in CUDA


We are interested in exploiting as much parallelism as possible, but the
ability to express nested maps leaves us in a dilemma: Which of the nested maps
should be run in parallel?

Various ways exist to treat the problem of nested parallelism, with the
extremes being disallowing nested parallelism at all and the vectorisation
transformation of NESL \cite{nesl}. While the vectorisation transformation
removes issues of both nesting and irregularity, it may also incur a certain
amount of overhead that can be prohibitive to its practical application, see
\cite{Catanzaro2011}.

Nikola itself is able to handle some degree of nested parallelism by exploiting
the up to three-dimensional grid layout of CUDA threads.

We propose a different compromise. Based on viewing nikola programs a sequence
of parallel actions, that are themselves composed only of sequential parts, we
propose adding a primitive construct to Nikola that serves as a separator
between what should be executed sequentially and what should be executed in
parallel. The innermost potential parallel operation enclosing the expression
with the marker gets to be executed in parallel. We have chosen to name this
construct \texttt{sequential}.

Consider how the expression \texttt{map (\bs -> map f bs) as} (ignoring the
technical details of shape specification for clarity) may be given a parallel
execution semantics:

\begin{verbatim}

ex1 = map (\bs -> map (sequential . f) bs) as

ex2 = map (\bs -> sequential $ map f bs) as

ex3 = map (\bs -> map f bs) as

\end{verbatim}

In \texttt{ex1}, the outer map happens sequentially, while the inner map, being
the innermost potentially parallel operation outside \texttt{sequential} is
executed in parallel, while each application of function \texttt{f} happens
sequentially.

In \texttt{ex2}, the outer map is the innermost potentially parallel operation
enclosing \texttt{sequential}, and thus it is executed in parallel, while the
inner map of \texttt{f} is executed sequentially.

In \texttt{ex3}, either the inner map is executed in parallel if \texttt{f}
contains no potential parallelism. Otherwise some part of \texttt{f} gets to be
executed in parallel.

Which of the above is the more beneficial depends entirely on the precise
operations denoted by \texttt{f} in conjunction with the capabilities of the
hardware that is to execute them, and of course the shape of \texttt{as}.

\section{Implications for fusion}

Directing the parallel execution strategy using \texttt{sequential} interacts
loop fusion.  In each of the examples \texttt{ex1}-\texttt{ex3} above, some
parts of \texttt{as} may be computed in parallel, namely the hyperplane slices
corresponding with a parallel map.

But consider the example:
\begin{verbatim}
ex4 = map (\bs -> map f bs) (sequential as)
\end{verbatim}

Here we would require that both maps and \texttt{f} be executed sequentially,
at least down until an eventual point where \texttt{f} acts only on single
elements of \texttt{as}.

Obviously, we realise that these examples only give a small hint as to the
consequences of introducing a construct such as \texttt{sequential}.
Investigating this would be an obligatory part of further research on the
viability of \texttt{sequential}.

\section{Implementation considerations}

So far we have only seriously considered making \texttt{sequential} an
additional primitive, usable as a function of type \texttt{Exp t a -> Exp t a}.

However, inspired by the use of array representation type tags, we have also
entertained the idea that the execution mode dictated by use of
\texttt{sequential} could be mirrored in the type of terms. But as the types
used in Nikola terms are already a complicated, complete with interaction with
various typeclasses and the value lifting machinery, we have postponed further
exploration of this.

\section{Implementation status}

Currently our effort concerning the addition of \texttt{sequential} to Nikola
has been mostly hypothetical. The idea arose as a simple and versatile
alternative to full program vectorisation, and we think it holds some promising
in that respect. At least it appears worthy of further study.  While we have
made some effort at implementing it in Nikola, this is far from complete.
