\chapter{Directing parallelism}
\label{chap:directing-parallelism}
% \item Discussion
%   \begin{itemize}
%   \item Presentation of alternatives
%     \begin{itemize}
%     \item Full vectorisation
%     \item CUDA 5.0 nested calls, we should definitely read about how
%        NVIDIA tackles the problem of mapping nested programs.
%      \item Repa computeS/computeP
%      \item Type tagging
%     \end{itemize}
%   \item Allows for experimentation with different parallelization
%   strategies
%   \item Possible alternative to Repa computeS/computeP
%   \end{itemize}
% \end{itemize}

% * Motivation
%   - NDP
%   - Search for alternatives to full vectorisation
%   - Vectorisation overhead

% * What we have done and our progress with implementation

As presented in the survey, Chapter \ref{chap:expressiveness}, we have
found that flat data-parallelism in the style of Accelerate and Nikola
(without our mapNest extension) is a non-composable and unnatural program
structure that hides the intended meaning of a program in clutter.

To look for alternatives, we have taken a look into nested data-parallelism,
and how it can be implemented in a GPU language. We have contemplated an
implementation of the vectorisation transformation (see
\cite{blelloch1990vector}), but ended up opting for a solution that is more
lightweight and hopefully easier to understand the execution semantics (and
thus performance properties).

Currently our effort concerning the implementation of our proposal
into Nikola has been mostly hypothetical but we find it worthy of
further study.  While we have made some effort at implementing it in
Nikola, this is far from complete.

In the next section we will introduce our take on nested
data-parallelism, and in the remaining sections of the chapter we will
discuss the implications and alternatives to such an extension.

\section{The \texttt{sequential}-operator}
% * Presentation of sequential
%   - A few small examples
%   - Use-cases in our option pricing examples

Thus motivated, we will now take a look at a typical nested
data-parallel program, inspired by our binomial option pricing case,
and the employ our suggested extension, to direct how the
parallelization is performed.

Algorithm \ref{alg:binomial-algorithm} from Section
\ref{sec:binomial-model} can be written in functional style as the
function:
\begin{lstlisting}
binomial optionParameters = foldl (map f ...) bs [t..0]
\end{lstlisting}
Here we have left out quite a lot of detail, to keep focus on what is
important for our example, the structure of the program.

We have seen one way this can be parallelized by issuing the
\lstinline{map} in parallel and synchronize between each
iteration of the fold, but we will now look at the problem of
portfolio pricing, where several options are priced simultaneously. Do
to time constraint we have not studied this problem closely in the
survey, but it serves as a good illustration in the current
context. In this case, there are two possible parallization
strategies. We can either take one option at a time, and parallize in
the way just described:
\begin{lstlisting}
portfolio@$_1$@ = map binomial
\end{lstlisting}
Alternatively, we could issue the pricing of each option as a separate
thread, where the pricing of the individual option is a sequential algorithm:
\begin{lstlisting}
portfolio@$_2$@ = parmap binomial
\end{lstlisting}
where the \lstinline{map} of the the \lstinline{binomial} algorithm is
executed sequentially.

%We will get back to the block-level parallel solution from the CUDA SDK,
%Algorithm \ref{alg:cuda-binom}, Section \ref{sec:binomial-model}.
%\todo[noinline]{make sure that we return to this}.
For now we will look at the theoretical performance of these two
implementations.  Defining $t_i$ to be the duration time of option number $i$
in the portfolio consisting of $n$ options, here is the expected depth
complexities for each variant:

\begin{align*}
  \text{\lstinline{portfolio}}_1 &= \textrm{Depth}\left(\sum_{i=1}^{n} \text{\lstinline{t}}_i\right) \\
  \text{\lstinline{portfolio}}_2 &= \textrm{Depth}\left(\max_{i=1}^{n} \frac{\text{\lstinline{t}}_i(\text{\lstinline{t}}_i+1)}{2}\right) \\
\end{align*}

The first case, \lstinline[mathescape]{portfolio$_1$}, will plausibly be an
efficient strategy for few long-running options of various length, as there are
few inputs that each display good parallel resource usage. On the other hand
\lstinline[mathescape]{portfolio$_2$} will be more efficient for problems with
many options of shorter and similar length.

\subsection{Our proposal}
Now, what we propose is to add a primitive construct to Nikola that
serves as a separator between what should be executed sequentially and
what should be executed in parallel, the idea being that it should be
simple to move between different parallelization strategies  without
changing the structure of the code. We have chosen the name \lstinline{sequential}
for the construct that represents this division.

In the above example, we could use \lstinline{sequential} to write
\lstinline[mathescape]{portfolio$_2$}, without having both a \lstinline{parmap}
and a \lstinline{map}:
\begin{lstlisting}
portfolio@$_2$@ = map (sequential binomial)
\end{lstlisting}
which should be understood as ``convert \lstinline{binomial} to a sequential
function, whereby the outer \lstinline{map} can be executed in
parallel''. Such a conversion is of course possible, as all parallel
programs can be rewritten as a sequential program.

One should then think about a map as a ``potentially parallel'' operation,
and only after investigating the point of use can we determine
whether it is to be evaluated in parallel or sequentially. In this
case the inner \lstinline{map} ends up being evaluated sequentially
rather than parallel, the \lstinline{foldl} operation does not have a
parallel version, and thus only have sequential semantics.

Informally, the suggested parallel execution semantics of programs written using
\lstinline{sequential} is:
\begin{quote}
  \emph{The innermost potentially parallel operation enclosing the
    expression with the }\lstinline{sequential}\emph{-marker gets to
    be executed in parallel.}
\end{quote}

As mentioned, the \lstinline{foldl} in the example can not be executed
in parallel as there is a strict dependence between each iteration,
but there are still more than one possible execution strategy for the
reduction. If we look at the case of
\lstinline[mathescape]{portfolio$_1$} where there is no use of
\lstinline{sequential}, we have the semantics that the innermost
\lstinline{map} will be executed in parallel. Thus, the
\lstinline{foldl} will have to be a host function, that serves as
synchronization point between each parallel \lstinline{map}.

That is, our proposal has the influence that we merge the host and GPU code
into a single language. Depending on the placement of
\lstinline{sequential}-markers in the code, we get different theoretical
complexities.

We can view \lstinline{sequential} as a meta-language construct, which
in a preprocessing step converts our language into a lower level
language where array operations are materialized as concrete
implementations which are either: sequential device code, parallel
device code or sequential host code. We are not suggesting that this
is the optimal strategy for an actual implementation of a language
with \lstinline{sequential}.

Using Nikola or Accelerate, one is to write GPU code in the embedded
DSL and use standard Haskell constructs in all other cases. With our
proposal we unify host and device code of a problem into a single
language.


\subsection{Evaluation strategy}
% * Use the syntax tree as a dependency graph
% * Execute kernels which have met all their dependencies
% asynchronously.
% * Synchronize before a kernel launch
% * If a sequential expression is not enclosed by a parallel
%   array operator, the expression could be executed in a CPU thread
% *
There are several ways this can be mapped to a concrete machine with a
graphics card. We suggest that the runtime uses the syntax tree as a
an acyclic dependency graph, and for all kernels where all
dependencies are met, executes them concurrently and asynchronuously.

Synchronization with possible dependencies is then necessary before
any operation waiting can to be executed.

Executing the kernels asynchronously leaves CPU time for other tasks,
and we might be able to extend this to a heterogeneous language. See
Section \ref{sec:parallelism-future-work} below.

\section{Consequences and complications}
\label{sec:sequential-complications}
% zipWith f (sequential as) bs
% map (\b -> if b then sequential $ map f ys else map f ys) xs

Even though the semantics explained above might seem simple, which is
also one of our goals, there is still a lot of complications that we
have to be aware of. As an example, how does \lstinline{sequential}
interact with conditionals, as in the following example:
\begin{lstlisting}
map ( \b -> if b
            then sequential (map f ys)
            else map f ys)
\end{lstlisting}
We have to decide the parallel execution semantics of the outer
\lstinline{map} before we can launch it. In such cases we have to make
special cases for our otherwise simple rule above. One option could be
to make \lstinline{sequential} spill over to the
\lstinline{else}-branch, and the outer \lstinline{map} would be a
parallel map. Another option is that the outer map is executed as a
host function and depending on the value of \lstinline{b} we
execute the inner \lstinline{map} sequentially on the host version or
in parallel on the device.

A third alternative is to add restrictions such that the above code is
not possible to write. That is, such that each branch of the
sequential must located on the same execution level. This is perhaps
possible by adding the execution strategy to the type of expressions,
we will discuss this further below (see Section \ref{sec:parallelism-future-work})

Another problematic, but similar, example is functions of several arguments:
\begin{lstlisting}
zipWith f (sequential as) bs
\end{lstlisting}

Here \lstinline{zipWith} is forced to proceed sequentially, as the elements of
\lstinline{as} must be evaluated sequentially. However, any potential
parallelism in \lstinline{f} and \lstinline{bs} would still be exploitable,
and the two even fusable.

\subsection{The type of \texttt{sequential}}
% Exp t a -> Exp t a
% Array r sh a -> Array r sh a
In the Nikola context, it is worth looking at how
\lstinline{sequential} can be incorporated into the existing type
system. The semantics of the operation is purely about execution and
it thus has a type similar to identity function, with the restriction
of working on Nikola terms. One possibility is to give it the type
\begin{lstlisting}
sequential :: Array r sh a -> Array r sh a
\end{lstlisting}
but in this case we would not be able to write our
\lstinline[mathescape]{portfolio$_2$} as we did, as sequential would
not accept a function argument. Our code would instead be:
\begin{lstlisting}
portfolio@$_2$@ = map (sequential . binomial)
\end{lstlisting}
The only thing that all Nikola terms have in common is instances of typeclass
\lstinline{Reifiable a S.Exp}. So the most inclusive type would be
\lstinline{sequential :: Reifiable a S.Exp -> a -> a}. However, we do not take
for granted that a function of that signature with our intended purpose is
definable.

We have speculated about various ways to express a code marker such as
\lstinline{sequential}. Our initial idea was to add yet another type variable
to Nikola expressions, such that the "sequentiality" of a term would be encoded
in the type. The types of \lstinline{map}, \lstinline{zipWith} and so on should
then implement the semantics of switching mode in accordance with their
arguments.

While this idea appears to hold some merit, in the concrete case of Nikola we
made the judgement that the type system was already saturated with modelled
constraints, and thus we settled for an untyped version.
%\todo{We did have a better argument than ``it is complicated''}

%\subsection{Interactions with fusion}
% - Interactions with fusion
% - If we employ it as a meta-language construct, fusion can perhaps
% be performed as usual

\subsection{Possible runtime errors}
%   - How it will introduce possibilities for runtime errors (by
%     irregularity), and that we do not have a proposal for how to solve
%     this, and we are not that happy about that aspect.

As mentioned in the previous chapter, the introduction of nested maps
in the form suggested by our \lstinline{mapNest} Nikola-construct,
also introduces the possibility of runtime errors, as the suggested
contruct does not prevent irregular arrays, see Section \ref{sec:unfold-mapNest}.

Further problems such as this one might also arise for other
operations, when defining nestable versions.

In our mindset we have adopted the notion that seems to prevail especially in
high performance computing, that it is sometimes better to be complete with
regards to expressivity and potentially unsafe, rather than offering full
static safety but limited expression.

%\todo{plc: I somewhat fail to see the exact purpose of this subsection}
%\todo{hypothesize a solution and say that this is perhaps the greatest
%  obstacle for acceptance among other researchers, but we believe that
%  we should start with a language that can operate in the way that we
%  need, and then afterwards look for solutions to such problems. This
%  language can eventually serve as a lower level language for other
%  more targetted domain-specific languages where such problems has
%  been hidden}


\section{Extensions and future work}
\label{sec:parallelism-future-work}

We of course have a lot of future work, as the suggestions in this
chapter are highly hypothetical, in that we have not found time to
finish an actual implementation. There might (and we are almost sure
that there will) be more corner cases that we have not thought about,
than those mentioned in Section \ref{sec:sequential-complications}
above. In addition to the obvious task of making a real
implementation, we now look at a further set of extensions that might
be interesting to look closer at.

\subsection{Heterogeneous computing}
The idea of a cursor such as \lstinline{sequential} might extend to
other divisions between different types of parallelism in
heterogeneous computing. We might have a cursor to mark that something
should not be attempted to be parallelized on a GPU, but should be
executed on the CPU. Such a cursor might allow one to break out of our
language and back into the full Haskell language by exposing the
\lstinline{IO}-monad or more restricted monad, that would allow more
general computation than what we can provide on a GPU.

Another cursor might be used to indicate were we divide between block
level parallelism (where threads can synchronize with its peers
directly on the GPU) or full parallelism where the threads does not
have the possibility of synchronization on the the device. Such a
cursor might make it possible to write a version of the binomial
portfolio pricer similar to \textsc{GpuBinom} in Algorithm
\ref{alg:cuda-binom}.

A final idea would be cursors to allow tasks to be partitioned between
several devices, such as several graphics cards or a cluster.

Again, we have done a lot of speculation on these matters, but as we
do not have a concrete implementation, we can not conclude further on
their practicallity.

% cpuEval and more than one graphics card
% cloudEval


\subsection{Experimentation}
% - Allows for experimentation with different parallelization
%   strategies
% - Allows for automatic experimentation
Another interesting aspect of such a light-weight construct as
sequential, that only instructs how computation is mapped to the
hardware, is that we can experiment using different parallelization
strategies, just be moving it around in the code. This might even
allow for automatic experimentation of different parallelization
strategies on the current hardware, where \lstinline{sequential} automatically, such that it will not
even have to be exposed to the programmer.

One approach is to employ \emph{machine learning} on smaller training
runs to select between host or GPU execution as in the Qilin system
\cite{luk2009qilin}. Alternatively, an approach based on static
program analysis could be employed. Here it has also been suggested to
use machine learning to select whether a task is fit for GPU or CPU
execution, though using program features rather than training runs
\cite{grewe2011static}.

\subsection{Repa}
Repa also employs a notion of array manifestation in separate parallel and
sequential markers.

We find that \lstinline{sequential} might be able to replace both
\lstinline{computeS} and \lstinline{computeP}to give Repa a simpler
execution strategy with less space for runtime errors. As mentioned
previously, nested calls to \lstinline{computeP} will result in
runtime errors. Contrary to Nikola, we have speculated that it might
be possible to implement a \lstinline{sequential} construct as a type
parametrised monad, but this remains a speculation.

%\section{Alternatives to \texttt{sequential}}
%      * Full vectorisation
%      * CUDA 5.0 nested calls, we should definitely read about how
%        NVIDIA tackles the problem of mapping nested programs.
%      * Repa computeS/computeP


%%%%%%%%%%%%%%%%%%%%%%%%%%%%%%%%%%%%%%%

% \section{Comparison with CUDA programs}
% %   - How it allows you to write programs much in the same style as
% %     you are currently writing CUDA programs.

% If we take a step back, and look at programs written directly in CUDA
% C, they will all follow more or less the same pattern \todo{again
%   mention that CUDA 5.0 allows nesting, so newer programs might
%   operate entirely differently} of sequential host code interleaved by
% kernel launches, where each kernel contains sequential code.

% \begin{verbatim}
%   host code
%   ...
%   kernel0<<<blockSize, gridSize>>>(...)
%      kernel code
%      ...
%   host code
%   ...
%   kernel1<<<blockSize, gridSize>>>(...)
%      kernel code
%      ...
%   host code
% \end{verbatim}

% Each kernel can employ different memory patterns and perform different
% kinds of internal synchronization, while performing their work.

% When we compare this code to code written with sequential, we see that
% each kernel

%   - We believe that we can obtain comparable performance to code
%     written directly in CUDA


%%%%%%%%%%%%%%%%%%%%%%%%%%%%%%%%%%%%%%%%%%%%%%%%%%%%%%%%%%%%%%%%%%%

% We are interested in exploiting as much parallelism as possible, but the
% ability to express nested maps leaves us in a dilemma: Which of the nested maps
% should be run in parallel?

% Various ways exist to treat the problem of nested parallelism, with the
% extremes being disallowing nested parallelism at all and the vectorisation
% transformation of NESL \cite{nesl}. While the vectorisation transformation
% removes issues of both nesting and irregularity, it may also incur a certain
% amount of overhead that can be prohibitive to its practical application, see
% \cite{Catanzaro2011}.



% Nikola itself is able to handle some degree of nested parallelism by exploiting
% the up to three-dimensional grid layout of CUDA threads.

% We propose a different compromise. Based on viewing nikola programs a sequence
% of parallel actions, that are themselves composed only of sequential parts, we
% propose adding a primitive construct to Nikola that serves as a separator
% between what should be executed sequentially and what should be executed in
% parallel. The innermost potential parallel operation enclosing the expression
% with the marker gets to be executed in parallel. We have chosen to name this
% construct \lstinline{sequential}.

% Consider how the expression \lstinline{map (\bs -> map f bs) as} (ignoring the
% technical details of shape specification for clarity) may be given a parallel
% execution semantics:

% \begin{verbatim}

% ex1 = map (\bs -> map (sequential . f) bs) as

% ex2 = map (\bs -> sequential $ map f bs) as

% ex3 = map (\bs -> map f bs) as

% \end{verbatim}

% In \lstinline{ex1}, the outer map happens sequentially, while the inner map, being
% the innermost potentially parallel operation outside \lstinline{sequential} is
% executed in parallel, while each application of function \lstinline{f} happens
% sequentially.

% In \lstinline{ex2}, the outer map is the innermost potentially parallel operation
% enclosing \lstinline{sequential}, and thus it is executed in parallel, while the
% inner map of \lstinline{f} is executed sequentially.

% In \lstinline{ex3}, either the inner map is executed in parallel if \lstinline{f}
% contains no potential parallelism. Otherwise some part of \lstinline{f} gets to be
% executed in parallel.

% Which of the above is the more beneficial depends entirely on the precise
% operations denoted by \lstinline{f} in conjunction with the capabilities of the
% hardware that is to execute them, and of course the shape of \lstinline{as}.

% \section{Implications for fusion}

% Directing the parallel execution strategy using \lstinline{sequential} interacts
% loop fusion.  In each of the examples \lstinline{ex1}-\lstinline{ex3} above, some
% parts of \lstinline{as} may be computed in parallel, namely the hyperplane slices
% corresponding with a parallel map.

% But consider the example:
% \begin{verbatim}
% ex4 = map (\bs -> map f bs) (sequential as)
% \end{verbatim}

% Here we would require that both maps and \lstinline{f} be executed sequentially,
% at least down until an eventual point where \lstinline{f} acts only on single
% elements of \lstinline{as}.

% Obviously, we realise that these examples only give a small hint as to the
% consequences of introducing a construct such as \lstinline{sequential}.
% Investigating this would be an obligatory part of further research on the
% viability of \lstinline{sequential}.

% \section{Implementation considerations}

% So far we have only seriously considered making \lstinline{sequential} an
% additional primitive, usable as a function of type \lstinline{Exp t a -> Exp t a}.

% However, inspired by the use of array representation type tags, we have also
% entertained the idea that the execution mode dictated by use of
% \lstinline{sequential} could be mirrored in the type of terms. But as the types
% used in Nikola terms are already a complicated, complete with interaction with
% various typeclasses and the value lifting machinery, we have postponed further
% exploration of this.

% \section{Implementation status}

% Currently our effort concerning the addition of \lstinline{sequential} to Nikola
% has been mostly hypothetical. The idea arose as a simple and versatile
% alternative to full program vectorisation, and we think it holds some promising
% in that respect. At least it appears worthy of further study.  While we have
% made some effort at implementing it in Nikola, this is far from complete.

%%% Local Variables:
%%% mode: latex
%%% TeX-master: "../master"
%%% End:
