% Disposition for introduction:
% \begin{itemize}
% \item Where does performance and programmer efficiency come from
% \item The free lunch is over
% \item (Embarassingly parallel problems)
% \item Why the language-based approach?
% \item Parallel Functional Programming
% \item What is a "vector language"?
% \item Our strategy: implement real world example applications, study
%   limitations of current approaches, extend and modify until we get
%   better results.
% \item Report Outline
% \end{itemize}

\chapter{Introduction}
Ever since the first electronic computers were built, there has been a
wish for tackling larger and more complex problems, which in turn has
created an increasing demand for performance improvements and
increased programmer productivity. Performance improvements originate
from improvements in either hardware or the employed
algorithms. Programmer productivity, on the other hand, mostly
correlates with the features of the used programming language
\todo{cite}. A language providing high programmer efficiency should make
it easy to implement and reason about algorithms, make mistakes easily
avoidable and when a mistake happens, make it easy to uncover its
cause. \todo{cite}

For a handful of decades, we have been able to increase performance
through hardware improvements of sequential processors and we have
thus been able to stick with more or less the same model of
computation. Recently, hardware developers have faced physical
barriers, making further performance improvements of sequential
processors impractical, and they have had to go new ways to obtain the
desired speed-up \cite{sutter2006freelunchisover}. These new architectures call
for new algorithms and models of computation, as well as new languages
and programming tools to keep the complexity of software development
at a tolerable level.

We will focus on software development for \textit{graphics processing
  units} (GPUs), which were originally intended solely for computer
graphics, but have in recent years been found useful in many other
applications, such as \todo{Which applications?}. They have gained so
much popularity that they are a main ingredient in quite a few of
todays largest supercomputers.

Even though we think of graphics processors as \textit{modern
  hardware}, the software development tools in widespread use for
programming them are far from modern. CUDA and OpenCL, the two main
languages for programming GPUs are low-level languages with manual
memory management, limited abstractions, no \todo{critise
even more!}. Alternative approaches have been tried out, but none of them
have found as widespread use as CUDA and OpenCL. Such higher level
languages include Theano, Accelerate, Nikola, Obsidian, Intel Array
Building Blocks, Qilin and Copperhead.

\todo{describe the shared characteristica of these languages}
\todo{define parallel functional programming}
\todo{define ''vector language''}

We want to join in and contribute to these languages, and we believe
that such an endeavour should start with evaluating current state of
the art, such that we can make sure our contributions will be relevant
and beneficial. Thus, the first part of this report is a survey of
some of the existing parallel functional programming languages. We
have taken the strategy of implementing some real world example
applications from the financial domain, to study the limitations of
current approaches.

% PLC: Should we have these in the introduction? wouldn't they fit better in their respective sections?
\todo{Which kind of problems we have found}
\todo{Something about what we have done to alleviate those problems}

\section{Report outline}
The remainder of the report is structured as follows. This
introductory chapter is supplemented with two chapters introducing
terminology, one on the financial example problems we are going to use
throughout the report and one on hardware platforms such as GPUs. The
rest of the report after these introductory chapters is divided into
two independent parts. The first part contains a survey of currently
used vector languages and their implementation techniques. The second
part describes our own efforts into \todo{\ldots}

%%% Local Variables:
%%% mode: latex
%%% TeX-master: "master"
%%% End:
