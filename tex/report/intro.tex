% Disposition for introduction:
% \begin{itemize}
% \item Where does performance and programmer efficiency come from
% \item The free lunch is over
% \item (Embarassingly parallel problems)
% \item Why the language-based approach?
% \item Parallel Functional Programming
% \item What is a "vector language"?
% \item Our strategy: implement real world example applications, study
%   limitations of current approaches, extend and modify until we get
%   better results.
% \item Report Outline
% \end{itemize}

\chapter{Introduction}
\section{Background}
Ever since the first electronic computers were built, there has been a
wish for tackling larger and more complex problems, which in turn has
created an increasing demand for performance improvements and
increased programmer productivity. Performance improvements originate
from improvements in either hardware or the employed
algorithms. Programmer productivity, on the other hand, correlates
with the features of the used programming language and programming
environment (IDEs, debuggers, etc.). A language providing high
programmer efficiency should make it easy to implement and reason
about algorithms, make mistakes easily avoidable and when a mistake
happens, make it easy to uncover its cause.

For a handful of decades, we have been able to increase performance
through hardware improvements of sequential processors and we have
thus been able to stick with more or less the same model of
computation. Recently, hardware developers have faced physical
barriers, making further performance improvements of sequential
processors impractical, and they have had to go new ways to obtain the
desired speed-up \cite{sutter2006freelunchisover}. These new
architectures call for new algorithms and models of computation, as
well as new languages and programming tools to keep the complexity of
software development at a tolerable level.

We will focus on software development for \emph{graphics processing
  units} (GPUs), which in recent years have been found cost-effective
alternatives to ordinary CPUs, for many other domains than computer
graphics. Today they are employed in fields as diverse as
bioinformatics, computational chemistry, medical image analysis,
relational databases, computational finance and simulation in both
engineering and the sciences \cite{hwy2011emerald, hwu2011jade,
  owens2007survey}. Solutions to many problems in these domains can be
expressed naturally as \emph{data-parallel} algorithms, where each
data item can be processed independently (often from a linear algebra
specification of the problem). This is thus where GPUs popularity
arise; executing such data-parallel tasks is the core capability of
GPUs.

% Should we include this somewhere?
% GPUs have gained so much popularity that they are a main ingredient in
% quite a few of todays largest supercomputers \cite{Top500}.

\section{Data-parallel programming languages}
Even though we think of graphics processors as \emph{modern
  hardware}, the software development tools in widespread use for
programming them are far from modern. CUDA
\cite{nvidia2012programming} and OpenCL \cite{munshi2011opencl}, the
two main languages for programming GPUs are low-level languages with
manual memory management, limited abstractions, and tedious hand
optimization are often required. It is, for instance, important to
streamline memory accesses to get decent performance. The low-level
nature of these languages makes automated code analysis difficult. The
NVIDIA CUDA compiler thus cannot make an optimization such as stream
fusion (deforestation), that removes intermediate data structures and
thus removes the need for very costly memory transactions. The
programmer thus has to fuse programs by hand, sacrificing clarity as
abstractions between individual parts has to be broken down and fused
together. Also, the programmer has to be well aware of the exact
hardware his program is to be executed on, thus making a tight
coupling between program and hardware.

High-level languages for GPU programming have been developed and
employed, but none of them have found as widespread use as programming
directly in CUDA or OpenCL. These newer languages include
Theano\cite{bergstra2010theano},
Accelerate\cite{chakravarty2011accelerating},
Nikola\cite{mainland2010nikola}, Obsidian\cite{svensson2011obsidian},
Intel Array Building Blocks \cite{newburn2011intel}, Bohrium
\cite{homepage:bohrium} and Copperhead \cite{Catanzaro2011}. Their
common aim is to abstract away the underlying hardware platform and
let data-parallel problems be easily specified.

These languages share the characteristica of lifting operations to
work collectively on vectors rather than individually on primitive
values (such as integers and floating point number).

We want to join in and contribute to these languages, and we believe
that such an endeavour should start with evaluating current state of
the art, such that we can make sure our contributions will be relevant
and beneficial. Thus, the first part of this report is a survey of
some of the existing parallel functional programming languages. We
have taken the strategy of implementing some real world example
applications from the financial domain, to study the limitations of
current approaches.

% \todo{Add note on mapping nested data-parallelism to flat
%   architectures is still an open problem, as vectorisation is not the
%   holy grail}

\section{Contributions}

We have conducted a small survey comparing a few Haskell libraries and
languages with respect to their performance, expressivity and the apparent
health of their supporting projects infrastructure.

Taking an outset in two common financial algorithms for option pricing
and a quasi-random number generator, we have found that the lack of
nesting in a language such as Accelerate and Nikola is too limiting to
implement concise solutions and that the specification of some
algorithms would lead to incomposable contructs.

As this lack of nesting is due to excessive and concious limitation,
we suggest taking a step back and give the programmer more control,
though giving away some of the guarantees made by a strict type system
of performance and absolute correctness. Of our surveyed languages,
Nikola fits this view the best, and is selected as the subject for
further study.

This we go about by exploring the implementation of a few but
motivated extra array operators, and last we present a proposal to
deal with the problem posed by mapping nested array
operators\footnote{We avoid the term nested data-parallelism, as that
  is often connotated with the vectorisation transformation} to flat
parallel hardware. This proposal uses a cursor to mark the limit
between sequential and parallel code, and we find at promising
alternative to vectorisation. There is a large body of future work in
deducing the consequences of such an addition.

% \begin{itemize}

% \item A survey comparing of the vector languages CUDA, R, Repa, Accelerate and
%   Nikola. The comparison is based on how well they performed with our cases,
%   both in terms of expressivity and performance, and the surrounding community
%   of developers and users.

% \item We found Nikola to be the most promising candidate for useful extensions.

% \item We have implemented prototypes of folding, unfolding and nested maps in
%   Nikola. While there is yet some way to an acceptable implementation, we
%   outline a way to it.

% \item We present an alternative way of expressing nested data parallelism, that
%   doesn't involve an elaborate code transformation like \cite{nesl}, and
%   doesn't impair the compositionality of the language. Like above, we have yet
%   to actually test the method in practise.

% \end{itemize}

%
%* Example where the Accelerate breaks down, and code does not seem to
%  be written clearly
%
%* An alternative approach to specifying

All the source code associated with the practical execution of the survey, as well as this report is hosted at
\url{https://github.com/HIPERFIT/vectorprogramming/}.
The extensions developed for Nikola are to be found at \url{https://github.com/HIPERFIT/nikola/}.

\section{Report outline}
The remainder of the report is structured as follows. This
introductory chapter is supplemented with two chapters introducing
terminology: one on hardware platforms such as GPUs and one on the
financial example problems we are going to use throughout the
report. 

The rest of the report after these introductory chapters is divided
into two more or less independent parts. The first part contains a
survey of currently used data-parallel languages. The second part
describes our own efforts into extending the GPU language Nikola with
functionality we found lacking in the experiments conducted throughout
the survey.

% \todo{Use refs to chapters and parts in outline}

%%% Local Variables:
%%% mode: latex
%%% TeX-master: "master"
%%% End:
